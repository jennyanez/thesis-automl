\chapter{Propuesta de modificación al componente de AutoML para pre-procesado}\label{chap:2}



\section{Subcomponente para la discretización}
\subsection{Descripción del subcomponente}
\subsection{Requisitos y restricciones del subcomponente}
\subsection{Modelación del subcomponente}




\section{Componente para la optimización de hiperparámetros}
El componente propuesto, presente en la figura \ref{fig:automl-componente-hpo}, para la optimización de hiperparámetros, contiene la siguiente configuración:
\begin{figure}[H]
	\centering
	\includegraphics[width=0.35\linewidth]{"figuras/capi 2/automl-componente-hpo"}
	\caption[Componente AutoML Clasificación (Optimización de Hiperparámetros)]{Componente AutoML Clasificación (Optimización de Hiperparámetros)}
	\label{fig:automl-componente-hpo}
\end{figure}
\begin{enumerate}
	\item Puerto de entrada: recibe los datos de entrada en formato tabular.
	\item Elementos de la configuración:
	\begin{itemize}
		\item Columna objetivo: presenta las columnas de tipos String que pueden fungir como Columna objetivo.
		\item Estrategia de optimización de hiperparámetros: se selecciona la estrategia de optimización de hiperparámetros entre las disponibles (disponibles: Random Search, Bayesian Optimization (TPE), Brute Force y Hillclimbing).
		\item Selección del número de subconjuntos de la validación cruzada: el valor introducido determina la cantidad de veces que se divide el conjunto de datos en subconjuntos de entrenamiento y prueba durante el proceso de validación.
	\end{itemize}
	\item Puerto de salida: tabla de hiperparámetros optimizados siguiendo la estrategia de optimización escogida.
\end{enumerate}

\subsection{Uso y configuración del componente AutoML Clasificación (Optimización de Hiperparámetros)}
A continuación, se define la secuencia de pasos para el correcto funcionamiento del componente AutoML Clasificación (Optimización de Hiperparámetros):

\begin{enumerate}
	\item Proporcionar el conjunto de datos al puerto de entrada (el componente marcará error en este punto, pues la configuración es obligatoria).
	\item Dar click derecho sobre el componente, seleccionar “Configure…”. 
	\item Seleccionar la configuración deseada (como se observa en el ejemplo de la figura \ref{fig:config-automl-hpo}). 
	\item Ejecutar el componente (Inicialmente el puerto de salida se encuentra vacío).
	\item Dar click derecho sobre el componente y seleccionar “Interactive View: AutoML” (Fig3). 
\end{enumerate}

\begin{figure}[H]
	\centering
	\includegraphics[width=0.5\linewidth]{"figuras/capi 2/config-automl-hpo"}
	\caption[Ejemplo de configuración del componente AutoML (Optimización de Hiperparámetros)]{Ejemplo de configuración del componente AutoML (Optimización de Hiperparámetros)}
	\label{fig:config-automl-hpo}
\end{figure}

\begin{figure}[H]
	\centering
	\includegraphics[width=0.5\linewidth]{"figuras/capi 2/automl-hpo-vista-salida"}
	\caption[Vista de la salida.]{Vista de la salida.}
	\label{fig:automl-hpo-vista-salida}
\end{figure}

\subsection{Requisitos y restricciones del componente AutoML Clasificación (Optimización de Hiperparámetros)}
El componente propuesto debe cumplir los siguientes requisitos funcionales:

\begin{itemize}
	\item RF1: Seleccionar columna objetivo
	\item RF2: Seleccionar estrategia de optimización de hiperparámetros.
	\item RF3: Seleccionar cantidad de subconjuntos en la validación cruzada.
	\item RF4: Optimizar hiperparámetros y entrenar datos para Redes Neuronales de Retro propagación.
	\item RF5: Retornar tabla de hiperparámetros optimizados.
\end{itemize}

El componente propuesto presenta las siguientes restricciones para su funcionamiento:

\begin{itemize}
	\item Los datos de entrada deben encontrarse en formato tabular.
	\item La columna objetivo debe ser de tipo String.
	\item Los datos de entrada deben ser de tipo numérico, menos la columna objetivo.
	\item Los datos numéricos deben estar previamente normalizados.
\end{itemize}

\subsection{Modelación del componente AutoML Clasificación (Optimización de Hiperparámetros)}
El diagrama de flujo de la figura \ref{fig:diagrama-flujo-gral-comp-hpo} expone el flujo general del componente AutoML Clasificación (Optimización de Hiperparámetros).

\begin{figure}[H]
	\centering
	\includegraphics[width=0.7\linewidth]{"figuras/capi 2/diagrama-flujo-gral-comp-hpo"}
	\caption[Diagrama de flujo general del componente AutoML Clasificación (Optimización de Hiperparámetros)]{Diagrama de flujo general del componente AutoML Clasificación (Optimización de Hiperparámetros)}
	\label{fig:diagrama-flujo-gral-comp-hpo}
\end{figure}

La configuración de la selección de parámetros es el primer paso en el cual se crearán las variables que dictarán el comportamiento del flujo. Posteriormente se ejecuta la optimización de hiperparámetros para RProp, donde recoge los resultados en una tabla y luego los grafica.\\
El flujo KNIME correspondiente se evidencia en el Anexo \ref{aped.flujo-hpo-rprop}.

\subsubsection{Selección de parámetros}
Los parámetros que rigen el funcionamiento del componente propuesto brindan al usuario una mayor personalización de la optimización de hiperparámetros, pues le ofrece la libertad de configurar múltiples factores claves de esta etapa.\\
El diagrama de actividades de la figura \ref{fig:diagrama-act-selecc-param-hpo} expone el flujo para la selección de parámetros.

\begin{figure}[H]
	\centering
	\includegraphics[width=0.7\linewidth]{"figuras/capi 2/diagrama-act-selecc-param-hpo"}
	\caption[Diagrama de actividades de selección de parámetros]{Diagrama de actividades de selección de parámetros}
	\label{fig:diagrama-act-selecc-param-hpo}
\end{figure}

La selección de parámetros se realiza con los siguientes nodos de configuración, presentes en el repositorio base:

\begin{itemize}
	\item Seleccionar columna objetivo: se emplea el nodo Column Selection Configuration (Figura \ref{fig:nodo-column-select-conf}), el cual recibe una tabla y devuelve el nombre de la columna seleccionada como variable de flujo. En este caso presenta la configuración adicional para solo mostrar las columnas de tipo String.
	\begin{figure}[H]
		\centering
		\includegraphics[width=0.15\linewidth]{"figuras/capi 2/nodo-column-select-conf"}
		\caption[Nodo Column Selection Configuration]{Nodo Column Selection Configuration}
		\label{fig:nodo-column-select-conf}
	\end{figure}
	
	\item Seleccionar estrategia de optimización de hiperparámetros: la selección de la estrategia se lleva a cabo empleando el nodo Single Selection Configuration (Figura \ref{fig:nodo-single-select-conf}). Devuelve la variable strategy con el valor seleccionado previamente.
	\begin{figure}[H]
		\centering
		\includegraphics[width=0.15\linewidth]{"figuras/capi 2/nodo-single-select-conf"}
		\caption[Nodo Single Selection Configuration]{Nodo Single Selection Configuration}
		\label{fig:nodo-single-select-conf}
	\end{figure}
	
	\item Seleccionar número de subconjuntos en la validación cruzada: la selección de la cantidad de subconjuntos de partición de entrenamiento se lleva a cabo con el nodo Interger Configuration(Figura \ref{fig:nodo-int-conf}). El cual devuelve la variable de flujo resultante de la selección, en este caso presenta la configuración para limitar el rango entre 5 y 10.
\end{itemize}
\begin{figure}[H]
	\centering
	\includegraphics[width=0.15\linewidth]{"figuras/capi 2/nodo-int-conf"}
	\caption[Nodo Integer Configuration]{Nodo Integer Configuration}
	\label{fig:nodo-int-conf}
\end{figure}

\subsection{Optimización de hiperparámetros para Rprop}
Las Redes Neuronales por retro-propagación necesitan que todos los valores sean de tipo numéricos normalizados. Para el entrenamiento y prueba de las Redes Neuronales por retro propagación se emplean los nodos RProp MLP Learner y MultiLayerPerceptron Predictor (Figura \ref{fig:nodos-rprop}) respectivamente.

\begin{figure}[H]
	\centering
	\includegraphics[width=0.4\linewidth]{"figuras/capi 2/nodos-rprop"}
	\caption[Nodos para entrenar y probar Redes Neuronales por retro propagación]{Nodos para entrenar y probar Redes Neuronales por retro-propagación}
	\label{fig:nodos-rprop}
\end{figure}

El diagrama de actividades de la figura \ref{fig:diagrama-act-proc-rprop-hpo}, expone el flujo para el procesamiento necesario para la ejecución del algoritmo Redes Neuronales por retro-ropagación con optimización de hiperparámetros.
\begin{figure}[H]
	\centering
	\includegraphics[width=0.7\linewidth]{"figuras/capi 2/diagrama-act-proc-rprop-hpo"}
	\caption[Diagrama de actividades para el procesamiento de RProp con HPO]{Diagrama de actividades para el procesamiento de RProp con HPO}
	\label{fig:diagrama-act-proc-rprop-hpo}
\end{figure}

Para llevar a cabo el procesamiento RProp se emplean los siguientes nodos: 
\begin{itemize}
	\item Ciclo para recorrer el rango de hiperparámetros: se emplean los nodos Parameter Optimization Loop Start y Parameter Optimization Loop End (Figura \ref{fig:nodos-param-opt-loop}), estos dos nodos permiten guardar todas las iteraciones realizadas por el algoritmo con las diferentes combinaciones de hiperparámetros. Para su configuración se eligieron como hiperparámetros el número de capas, la cantidad de neuronas y el número máximo de iteraciones (Figura \ref{fig:conf-nodo-param-loop}).
	\begin{figure}[H]
		\centering
		\includegraphics[width=0.5\linewidth]{"figuras/capi 2/nodos-param-opt-loop"}
		\caption[Nodos Parameter Optimization Loop Start y Parameter Optimization Loop End]{Nodos Parameter Optimization Loop Start y Parameter Optimization Loop End}
		\label{fig:nodos-param-opt-loop}
	\end{figure}

	\begin{figure}[H]
		\centering
		\includegraphics[width=0.6\linewidth]{"figuras/capi 2/conf-nodo-param-loop"}
		\caption[Configuración del nodo Parameter Optimization Loop Start]{Configuración del nodo Parameter Optimization Loop Start}
		\label{fig:conf-nodo-param-loop}
	\end{figure}
	
	\item Ciclo para dividir el conjunto de datos: se emplean los nodos X-Partitioner y X-Aggregator (Figura \ref{fig:nodos-cross-val}), para dividir el conjunto de datos en K particiones para realizar una validación cruzada, donde cada partición se utiliza como conjunto de prueba una vez y las otras K-1 particiones se utilizan como conjunto de entrenamiento.
	\begin{figure}[H]
		\centering
		\includegraphics[width=0.4\linewidth]{"figuras/capi 2/nodos-cross-val"}
		\caption[Nodos X-Partitioner y X-Aggregator]{Nodos X-Partitioner y X-Aggregator}
		\label{fig:nodos-cross-val}
	\end{figure}

	\item Calcular la exactitud: se emplea el nodo Scorer (Figura \ref{fig:nodo-scorer}), el cual recibe la predicción y la columna objetivo en una tabla para la evaluación. 
	\begin{figure}[H]
		\centering
		\includegraphics[width=0.15\linewidth]{"figuras/capi 2/nodo-scorer"}
		\caption[Nodo Scorer]{Nodo Scorer}
		\label{fig:nodo-scorer}
	\end{figure}
	
	\item Graficar: se emplea el nodo Table View (Fig13) capaz de visualizar la tabla de los hiperparámetros de mejor resultado.
	\begin{figure}[H]
		\centering
		\includegraphics[width=0.15\linewidth]{"figuras/capi 2/nodo-table-view"}
		\caption[Nodo Table View]{Nodo Table View}
		\label{fig:nodo-table-view}
	\end{figure}
	
\end{itemize}

\section{Conclusiones parciales}
\pagebreak

