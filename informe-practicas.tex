\documentclass[12pt,letterpaper]{report}

\usepackage[utf8]{inputenc}
\usepackage[spanish]{babel}
\usepackage{amsmath}
\usepackage{amsfonts}
\usepackage{amssymb}
\usepackage{float}
\usepackage{makeidx} 
\usepackage{graphicx}
\usepackage{color}
%\usepackage{kpfonts} % change font
%\usepackage{mathptmx} % change font
\usepackage{XCharter} % Use the XCharter fonts
\usepackage{multirow}
%\usepackage{natbib}
%\usepackage[numbers]{natbib}
\usepackage[authoryear,comma]{natbib}
%\usepackage{amsmath,amssymb,amsfonts,latexsym,cancel}
\usepackage[left=3cm,right=3cm,top=2cm,bottom=2cm]{geometry}
\usepackage{subcaption}
%\usepackage{url}
\usepackage[breaklinks,colorlinks=true,linkcolor=black,citecolor=black,urlcolor=black]{hyperref}
\usepackage{graphicx}
\usepackage{makecell}
\usepackage{enumitem, hyperref}
\usepackage[table]{xcolor}
\usepackage{longtable}
\usepackage{booktabs}
\usepackage{appendix}
\usepackage{rotating}
\usepackage[spanish,onelanguage,ruled,vlined,lined,resetcount]{algorithm2e}
\SetKw{Break}{finalizar ciclo;}
\usepackage[paper=portrait, pagesize]{typearea}
\usepackage{fancyhdr} % para poner el título en cada capitulo
\usepackage{rotating} % para rotar las paginas
\usepackage{pdflscape}
\usepackage{tablefootnote}
\usepackage{setspace}
\usepackage{comment}
\usepackage{array}

 
\usepackage{vmargin}
\setmargins {2.5cm}       % margen izquierdo
{1.5cm}                        % margen superior
{16.5cm}                      % anchura del texto
{23.42cm}                    % altura del texto
{10pt}                           % altura de los encabezados
{1cm}                           % espacio entre el texto y 			los encabezados
{0pt}                             % altura del pie de página
{2cm} 

% genera el mes de forma automatica
\usepackage{datetime}
\makeatletter
\newdateformat{mifecha}{La Habana, \monthname[\THEMONTH] \THEYEAR}
\renewcommand{\monthnamespanish}[1][\month]{
	\@orgargctr=#1\relax
	\ifcase\@orgargctr
	\PackageError{datetime}{Invalid Month number \the\@orgargctr}{
		Month number should go from 1 to 12
		}
		\or Enero
		\or Febrero
		\or Marzo
		\or Abril
		\or Mayo
		\or Junio
		\or Julio
		\or Agosto
		\or Septiembre
		\or Octubre
		\or Noviembre
		\or Diciembre
		\else \PackageError{datetime}{Invalid Month number \the\@orgargctr}{
			Month number should go from 1 to 12
		}
		\fi
		}


% permite crear listas dentro de lasa tablas
\newcolumntype{e}[1]{%--- Enumerated cells ---
	>{\minipage[t]{\linewidth}%
		\NoHyper%                Hyperref adds a vertical space
		\let\\\tabularnewline
		\enumerate
		\addtolength{\rightskip}{0pt plus 50pt}% for raggedright
		\setlength{\itemsep}{-\parsep}}%
	p{#1}%
	<{\@finalstrut\@arstrutbox\endenumerate
		\endNoHyper
		\endminipage}}

\newcolumntype{i}[1]{%--- Itemized cells ---
	>{\minipage[t]{\linewidth}%
		\let\\\tabularnewline
		\itemize
		\addtolength{\rightskip}{0pt plus 50pt}%
		\setlength{\itemsep}{-\parsep}}%
	p{#1}%
	<{\@finalstrut\@arstrutbox\enditemize\endminipage}}
\makeatother

% esto es para que funcionen los anexos incluyendo la ñ
\makeatletter \renewcommand*{\numberline}[1]{% 
	\hb@xt@\@tempdima{% 
		#1% 
		\protected@edef\@temp@num{#1}% 
		\ifx\@temp@num\@empty\else .\fi \hfil }% 
	} \makeatother


\newenvironment{dedication}
{\clearpage           % we want a new page
	\thispagestyle{empty}% no header and footer
	\vspace*{\stretch{.5}}% some space at the top 
	\itshape             % the text is in italics
	\raggedleft          % flush to the right margin
}
{\par % end the paragraph
	\vspace{\stretch{3}} % space at bottom is three times that at the top
	\clearpage           % finish off the page
}


\title{Nueva versión de componente KNIME para AutoML en tareas de clasificación}

\makeindex 

\renewcommand{\baselinestretch}{1.2}

\begin{document}
	\pagenumbering{roman}	
	\renewcommand{\listtablename}{Índice de tablas}
	\renewcommand{\tablename}{Tabla}
	\renewcommand{\bibname}{Referencias bibliográficas}
	
	\renewcommand{\appendixname}{Anexos}
	\renewcommand{\appendixtocname}{Anexos}
	\renewcommand{\appendixpagename}{Anexos}
	
%	esto es para que las listas anidadas salgan con numeros
	\renewcommand{\labelenumii}{\arabic{enumi}.\arabic{enumii}}
	\renewcommand{\labelenumiii}{\arabic{enumi}.\arabic{enumii}.\arabic{enumiii}}
	\renewcommand{\labelenumiv}{\arabic{enumi}.\arabic{enumii}.\arabic{enumiii}.\arabic{enumiv}}
	
	
	\pagestyle{empty}	
	\markboth{}{} 	
	\thispagestyle{empty} 
	
 	\begin{figure}
	\centering
	\includegraphics[width=0.5\textwidth]{figuras/membrete-cujae-centrado.png}
\end{figure}

	\begin{center}
		
	\vspace{2cm}

	
	
	\large{Informe de prácticas profesionales}

	\vspace{3cm}
	

	\LARGE{\textbf{Modificación de componente KNIME de AutoML en tareas de clasificación.}}
	
	
	
	\vspace{2cm}
	
	\large{
	\textbf{Autores:} \\
	Jennifer Yanez Jiménez\\
	Rainer Pellerano Alvarez\\
	\vspace{1.5cm}
	
	\textbf{Tutora:}\\
	Dra. Raisa Socorro Llanes\\
	}
	
	\vspace{2cm}
	
	{\mifecha\today}
	
\end{center}	



 	\section*{Declaración de autoría}

Lorem ipsum dolor sit amet, consectetur adipiscing elit, sed do eiusmod tempor incididunt ut labore et dolore magna aliqua. Facilisi etiam dignissim diam quis. Lacinia at quis risus sed. Eu feugiat pretium nibh ipsum consequat. Viverra mauris in aliquam sem fringilla ut. Aliquam ultrices sagittis orci a scelerisque. Erat imperdiet sed euismod nisi porta lorem. Libero id faucibus nisl tincidunt. Amet volutpat consequat mauris nunc congue nisi vitae. Tellus molestie nunc non blandit massa enim nec dui nunc. Leo vel fringilla est ullamcorper. Velit euismod in pellentesque massa placerat duis ultricies.
 	\thispagestyle{empty}
\begin{flushright}
	\begin{minipage}{12.5cm}
		\noindent
		\\[20em]
		%Modificar la cita que se quiere agregar
	%	{\Large Cita 01.}
	%	\\[3em]
	%	Autor
	%	\\ \textit{Fuente}
	%	\\[10em]
		%Para anular la adición de una segunda cita anule las siguientes lineas desde acá mediante comentario (%)
		\begin{flushright}
				\textit{
				“Y aunque el cuestionar me haga resbalar a la prudencia \\
				Sé que la duda es uno de los nombres de la inteligencia \\
				Mi semblante de estudiante es en esencia ser feliz \\
				Siendo el eterno postulante, el eterno aprendiz.”}
			\\[1em]
		\end{flushright}
		\begin{flushright}
			\textit{El Cuarteto de Nos}
		\end{flushright}
		
		%Hasta acá!
	\end{minipage}
\end{flushright} 




%	\section*{Opinión del tutor}
Lorem ipsum dolor sit amet, consectetur adipiscing elit, sed do eiusmod tempor incididunt ut labore et dolore magna aliqua. Cras tincidunt lobortis feugiat vivamus. Aliquam purus sit amet luctus venenatis lectus magna fringilla. Habitasse platea dictumst quisque sagittis purus sit amet. Vitae et leo duis ut diam quam nulla. Commodo quis imperdiet massa tincidunt. Amet tellus cras adipiscing enim eu turpis egestas. Faucibus ornare suspendisse sed nisi lacus sed. Facilisis leo vel fringilla est ullamcorper eget. Sagittis orci a scelerisque purus semper eget. Netus et malesuada fames ac turpis egestas maecenas pharetra convallis. Quis eleifend quam adipiscing vitae proin sagittis nisl. Enim sit amet venenatis urna cursus eget nunc scelerisque viverra. Ipsum dolor sit amet consectetur adipiscing elit. Ac odio tempor orci dapibus ultrices. Tincidunt vitae semper quis lectus nulla at volutpat diam. Maecenas accumsan lacus vel facilisis volutpat. Tristique sollicitudin nibh sit amet commodo nulla facilisi nullam vehicula. Quis risus sed vulputate odio. Pharetra vel turpis nunc eget lorem dolor sed viverra.

Hac habitasse platea dictumst vestibulum rhoncus est. Suspendisse faucibus interdum posuere lorem. Diam sollicitudin tempor id eu nisl nunc mi. Dignissim diam quis enim lobortis scelerisque. Sed tempus urna et pharetra pharetra massa massa ultricies mi. Urna duis convallis convallis tellus id interdum velit. Viverra ipsum nunc aliquet bibendum enim facilisis. Viverra accumsan in nisl nisi scelerisque eu. Facilisi nullam vehicula ipsum a arcu cursus vitae congue. Aliquet eget sit amet tellus cras adipiscing. Risus ultricies tristique nulla aliquet enim tortor at. Leo urna molestie at elementum eu. Mollis aliquam ut porttitor leo a. Convallis posuere morbi leo urna. Proin fermentum leo vel orci porta non pulvinar. Nulla facilisi morbi tempus iaculis urna id volutpat lacus. Pretium nibh ipsum consequat nisl vel pretium lectus quam id.
	\section*{Dedicatoria}
Lorem ipsum dolor sit amet, consectetur adipiscing elit, sed do eiusmod tempor incididunt ut labore et dolore magna aliqua. Cras tincidunt lobortis feugiat vivamus. Aliquam purus sit amet luctus venenatis lectus magna fringilla. Habitasse platea dictumst quisque sagittis purus sit amet. Vitae et leo duis ut diam quam nulla. Commodo quis imperdiet massa tincidunt. Amet tellus cras adipiscing enim eu turpis egestas.

Parturient montes nascetur ridiculus mus mauris vitae ultricies leo integer. Dolor morbi non arcu risus quis varius. Ac placerat vestibulum lectus mauris ultrices eros in. Interdum varius sit amet mattis. Volutpat sed cras ornare arcu dui vivamus. Vulputate dignissim suspendisse in est ante in nibh. Scelerisque eu ultrices vitae auctor eu augue ut. Potenti nullam ac tortor vitae purus faucibus ornare suspendisse sed. Massa eget egestas purus viverra. Viverra adipiscing at in tellus integer feugiat. Tellus molestie nunc non blandit. Augue interdum velit euismod in pellentesque massa placerat duis. Auctor elit sed vulputate mi sit amet mauris commodo quis. Pulvinar neque laoreet suspendisse interdum consectetur libero id faucibus nisl. Est ultricies integer quis auctor elit sed vulputate mi sit. Arcu vitae elementum curabitur vitae nunc sed velit. Nec nam aliquam sem et. Pellentesque elit eget gravida cum sociis natoque penatibus et. Viverra adipiscing at in tellus integer feugiat scelerisque. Porta lorem mollis aliquam ut porttitor leo a diam.

Aenean sed adipiscing diam donec adipiscing. Consectetur a erat nam at lectus urna. Cursus metus aliquam eleifend mi in nulla. Sed adipiscing diam donec adipiscing tristique risus nec. Molestie nunc non blandit massa enim nec dui nunc. Arcu vitae elementum curabitur vitae. Elementum integer enim neque volutpat ac. Nisi porta lorem mollis aliquam. Mi in nulla posuere sollicitudin aliquam ultrices. Rhoncus mattis rhoncus urna neque viverra justo nec ultrices. Sit amet tellus cras adipiscing enim eu turpis.
	\section*{Agradecimientos de Rainer}
A mi abuela, por ser una madre, muchas gracias, por todo el esfuerzo de criarme, nada fácil de seguro, este trabajo tiene mucho de ti, para no decir casi todo, que no paras de estar orgullosa, aunque no logre nada. Te quiero demasiado mamá. \\
A mi bisabuela, por tanto cariño que me ha dado, por demostrarme que, aunque ya las manos y los pies no den más, hay que seguir obrando, buscar el propósito de vivir. \\
A mi mamá, por el cariño y el amor que me ha dado todos los días, por haberme dado en herencia la parte burlona que me caracteriza, por confiar en mí, gracias. \\
A mi viejo, mi papá, por ser mi ejemplo de esfuerzo, trabajo duro y cuidado de la familia, gracias por apoyarme en todo momento. Dicen que los padres son los más exigentes, pero siempre me dejaste correr por la vida, apoyando mis ideas, mis decisiones, sin ponerme peros, gracias por estar, por guiarme, por quererme. \\
A mi abu, por ser mi referente profesional, por enseñarme que el ser ingeniero, es lo mejor del mundo y que el trabajo para llegar a ello nunca es fácil, por el amor y el cariño, por el tiempo que siempre tienes para cualquier conversación, gracias. \\
A Neni, por ser la que ha pasado todo el proceso a mi lado, en momentos buenos y malos, por darme el amor más grande que he recibido y aguantarme 24 x 24, que eso ya es un mérito, por todos estos años, por alimentarme jaja. Gracias, salchicha Lia. Te amo. \\
A todos los familiares que han formado parte de este proceso, gracias. \\
A Marco, mi hermano, por ser pieza clave en mi vida, que difícil se me hace separarme tanto de alguien que quiero, te extraño demasiado, pero tengo que estar feliz porque tu vida va a mejor, te quiero. La próxima vez que veas al bicho, dile que te firme una camiseta de Messi con las tres estrellas. \\
A Francito y Julito, por ser mi alegría, es verlos y sonrío, gracias por acompañarme todo este tiempo, gracias por los momentazos que siempre pasamos juntos, el cariño que siempre me han dado es impagable, les quiero mucho mis niños. \\
A Miguel, por ser la representación personificada de la fiesta. \\
A mis Guanajos Informáticos: Gordito, por ser lo mejor que he tenido en la carrera, uno aprobado y el otro suspenso no existía, nunca pensé encontrarme familia en la universidad y mira, la próxima carrera que hagamos espero que no esté tan fácil como esta y nos haga estudiar más, gracias por ser mi tándem, gracias por hacerme parte de tu vida (mención especial a la ex casa y a tu hermana que siempre estuvo ahí con nosotros, también te quiero). A Jani por ayudarnos tanto, no hubiese título sin ti, gracias por la paciencia, por la guía, por darnos luz en malos momentos, fuimos a re de numérica por tu culpa, gracias por ser parte fundamental en todo el proceso. A Roi por ser tan s…, tu sabe, siempre fuimos de final de champions. \\
A Jenny, que solo te puedo agradecer y agradecer, no me imaginé que esto fuera tan especial, llenaste algo en mí, que no sé qué es, pero me faltaba, me siento orgulloso de terminar esto a tu lado, me hiciste ser mejor, gracias por haberme acompañado de la mano, te quiero mucho y siempre recuerda \emph{baby, saca esa perra a pasear}. \\
A Raisa, por ser la mejor tutora sacando balones en la línea, gracias por el tiempo que nos dedicas y la sonrisa con la que siempre nos recibe. \\
A mis amigos de aula y a todos los que me ayudaron en algún momento, muchas gracias. \\


	\section*{Agradecimientos de Jennifer}
Seré breve, o eso espero. \\
A mi mamá, que supongo que nunca te he agradecido por todo lo que has hecho por mi, asumo que este es un buen momento para hacerlo. Gracias. Incluso debo agradecerte por Cloe, que ha sido mi fiel acompañante en mis noches de estudio, y en casi todo. Las quiero un mundo.\\
A mi papá que, a pesar de la distancia, ha estado ahí para mi en mis altos y bajos. Metafórica y literalmente, soy lo que soy gracias a ti. Gracias por guiarme, por siempre tener un consejo que darme, por confiar en mi cuando yo no lo hacía, por ser mi inspiración, mi héroe, en fin, gracias por todo. Te extraño un montón y te quiero más aún. \\
A Nelson, que ha sabido comprender mis frustraciones y siempre encuentra la forma de hacerme salir de ellas, ya sea tomando helado, sentándonos en el malecón o simplemente haciéndome reír. Gracias por estar siempre para mí, por siempre creer en mi, por tu cariño incondicional. Gracias por aguantarme, aunque digas lo contrario. Sinceramente, sin ti no sé que hubiera hecho.\\
A mis amistades de la facultad, que mencionaré de forma aleatoria sin poner un orden de importancia: \\
A Karlita, mi compañera desde la Lenin y asesora de moda, ojalá pudiéramos haber llegado hasta acá juntas, al igual que el Chichi, que aun le debo 5 pesos; \\
A Carlos, mi corazón de melón, compañero de excursiones y mi primer amigo de la universidad, gracias por soportarme en mi etapa más inmadura; \\
A Jeank, mi confidente, gracias por siempre estar ahí aguantando mis berrinches (que eran bastantes jajaja), tus consejos y tu sentido del humor han marcado la diferencia, gracias por quererme tanto; \\
A Ana, gracias por ser mi amiga, nunca pensé conocer a una persona que se pareciera tanto a mi, indudablemente ha sido un error en la Matrix;\\
A Ahmed, mi compañero de estudio desde primer año, estoy segura de que nunca vas a olvidar que \textsc{printf} va antes de \textsc{scanf}, en general, gracias por todo; \\
A Andy, mi tutorado y jevito de proyecto, la persona más noble que he conocido, es imposible no quererte; \\
A Erne, que sin ti no estuviera aquí, yo siempre molestándote para que me explicaras durante las clases online, y tu siempre tan dispuesto a hacerlo;\\
Al Cami, qué decir de ti, mi salud mental dependía de ti, con tus ocurrencias, consejos y recomendaciones. \\
Cada uno ha formado parte de mi historia en la uni, y sin ustedes, sin el granito de arena que han aportado a mi carrera y a mi vida en sí, nada sería igual. Y a los que no mencioné, perdonen, dije que sería breve, pero saben que los llevo en mi cora. \\
A Ray, porque por supuesto, un pequeño apartado es para ti. Gracias por todo tu apoyo, aunque sinceramente, no tengo palabras para agradecerte por todo. Eres, sin duda alguna, el mejor amigo que me puedo llevar de la universidad, el mejor compañero de tesis que alguien puede tener, mi salud mental 2.0 (Camilo estuvo primero jajaja). Sin ti, esta experiencia no habría sido lo mismo, y estoy casi segura de que me hubiera vuelto (más) loca hace bastante rato. Te quiero bro. \\
A los profesores de la facultad, que a pesar de todo, hicieron que me enamorara de mi carrera. \\
A mi tutora Raisa, es un honor para mi haber sido su tutorado. Gracias por siempre explicarnos todo con la paciencia que la caracteriza, por estar ahí para defendernos y apoyarnos en los tribunales donde por novatos creíamos que no íbamos a aprobar. \\
A todos los que hicieron posible que llegara hasta aquí, gracias. \\ \\
\textit{P.D.: Al final, no fui tan breve…}
	\section*{Resumen}
Vivimos en un mundo en el que se generan grandes cantidades de datos, los cuales se almacenan en diferentes sistemas, y el reto es convertir esos datos en información útil para la toma de decisiones. Una técnica para extraer información valiosa de grandes cantidades de datos es el Aprendizaje Automático, que se enfoca en el desarrollo de modelos y algoritmos que permiten a las computadoras aprender de los datos sin ser programadas explícitamente para hacerlo. Para implementar efectivamente estas técnicas, se requiere de la intervención humana, y la automatización del aprendizaje automático (AutoML) se ha desarrollado como una solución para simplificar y acelerar este proceso. Para ello existen numerosas herramientas, como KNIME, que permite la implementación de AutoML a través de diferentes nodos. En \citep{Carrazana2022} se desarrolla un componente dedicado a esta tarea, específicamente para el pre-procesado en tareas de clasificación. No obstante, quedaron tareas pendientes, como la optimización de hiperparámetros y la automatización de tareas en la fase de transformación de los datos, las cuales son implementadas en la presente investigación. 



\begin{description}
	\item[Palabras clave:]{Aprendizaje Automático, Minería de datos, AutoML, KNIME, optimización de hiperparámetros, preprocesamiento de datos, clasificación.}
\end{description}
%\end{abstract}



	\section*{Abstract}


\begin{description}
	\item[Keywords:]{}
\end{description}
%\end{abstract}


	

	\markboth{}{}
	\pagestyle{plain}	
	
	\tableofcontents	
	\pagebreak	
	
	\listoftables
	\pagebreak
	
	\listoffigures	
 	\clearpage 
	
	\pagebreak	
	\pagenumbering{arabic}
	
	\phantomsection
	\addcontentsline{toc}{chapter}{Introducción}
	\pagestyle{fancy} 
		
	\fancyhf{}
	\lhead[\thepage]{\textbf{Introducción}}
	\rhead[\textbf{Introducción}]{\thepage}
	
	\chapter*{Introducción}
En la actualidad, en el mundo se generan cada vez más datos y se almacenan en diversos tipos de sistemas, lo que nos presenta un gran desafío: ¿cómo convertir estos datos en información valiosa que pueda ser utilizada para tomar decisiones informadas? Para resolver este reto, se han desarrollado diversas técnicas y herramientas para extraer información útil de grandes cantidades de datos. \\
El proceso de descubrimiento de conocimiento en bases de datos (\textit{KDD}, por sus siglas en inglés), es un procedimiento que se utiliza para extraer conocimiento útil y relevante, a partir de grandes cantidades de datos almacenados en diversos sistemas \citep{orallo2004}. Este consta de varias fases: integración y recopilación; selección, limpieza y transformación; aplicación de algoritmos de minería de datos; evaluación e interpretación; así como la difusión y uso del conocimiento obtenido \citep{Han2011}. El proceso de descubrimiento de conocimiento en bases de datos ha sentado las bases para una disciplina relacionada, conocida como minería de datos. \\
La minería de datos es el proceso de extraer conocimiento útil y comprensible, previamente desconocido, desde grandes cantidades de datos almacenados en distintos formatos \citep{orallo2004}. Una de las técnicas más comunes de la minería de datos es la clasificación, que consiste en la identificación de un conjunto de categorías o etiquetas para un conjunto de datos no etiquetados \citep{orallo2004}. \\
La clasificación es una técnica muy utilizada en diferentes áreas, como la detección de spam en el correo electrónico \citep{mendez2007sistemas}, la clasificación de imágenes \citep{borras2017clasificacion} y la identificación de transacciones fraudulentas en tarjetas de crédito \citep{dhankhad2018supervised}. Al utilizar algoritmos de clasificación es posible predecir la categoría a la que pertenece un nuevo conjunto de datos, lo que puede ser de gran utilidad en la toma de decisiones y la mejora de los procesos. Esta tarea, a su vez, es uno de los principales enfoques del Aprendizaje Automático (Machine Learning), una disciplina dentro de la inteligencia artificial, que se basa en la idea de que las computadoras pueden aprender a reconocer patrones y tomar decisiones precisas y acertadas, a través de la experiencia y la retroalimentación continua de los datos. \\
El Aprendizaje Automático se enfoca en el desarrollo de algoritmos y modelos que permiten a las computadoras aprender de los datos sin ser programadas explícitamente para hacerlo. Se define como un conjunto de métodos que puede detectar patrones en los datos automáticamente, y luego usar los patrones descubiertos para predecir datos en el futuro, o para realizar otro tipo de toma de decisiones bajo incertidumbre \citep{murphy2012machine}. Estos patrones interesantes son los que representan el conocimiento. La implementación efectiva de estas técnicas requiere la intervención humana, incluyendo la selección de algoritmos adecuados, el pre-procesamiento de datos y la optimización de hiperparámetros. La Automatización del Aprendizaje Automático (AutoML) se ha desarrollado como una solución para simplificar y acelerar este proceso. \\
 AutoML tiene como objetivo tomar estas decisiones de una manera automatizada, objetiva y basada en datos: el usuario simplemente proporciona datos y el sistema AutoML determina automáticamente el enfoque que funciona mejor para esta aplicación en particular \citep{hutter2019automated}. Para realizar esta tarea de manera efectiva, se requiere de herramientas especializadas que permitan procesar grandes cantidades de información de forma rápida y eficiente. Una de estas herramientas es KNIME.\\
KNIME es una herramienta popular que proporciona un entorno de desarrollo visual para la creación, ejecución y evaluación de flujos de trabajo de análisis de datos. Es una plataforma de software libre y abierto que incluye una amplia variedad de nodos para el pre-procesado de datos, la minería de datos y la modelización de datos \citep{knime2023}. En KNIME, el AutoML se puede implementar a través de una extensión de H2O y dos componentes AutoML; la extensión de H2O, que acoge múltiples nodos para tareas concretas; y el componente \textit{AutoML Learner}, que permite seleccionar automáticamente el mejor algoritmo de aprendizaje automático para un conjunto de datos en particular, así como ajustar los hiperparámetros del modelo. No obstante, una de las principales desventajas que presenta es la falta de control y personalización. Aunque permite a los usuarios seleccionar automáticamente los algoritmos, no tienen control directo sobre estos procesos y no pueden ajustar los parámetros de manera manual. Esto puede limitar la capacidad de los usuarios para personalizar y optimizar los modelos para satisfacer sus necesidades específicas. Mientras, el componente \textit{AutoML (Componente AutoML Clasificación (pre-procesado)} \citep{Carrazana2022}, desarrollado en la CUJAE, ejecuta y compara el desempeño de múltiples flujos de AutoML en tareas de clasificación. En este componente se desarrollaron subcomponentes enfocados en tareas concisas de pre-procesado. Sin embargo, no contempla la optimización de hiperparámetros y la automatización de algunas de las actividades esenciales en el pre-procesado, como discretización y normalización, quedaron pendientes, lo que dificulta el procesamiento de datos y la precisión de los modelos de Aprendizaje Automático. De esta manera se genera un \textbf{problema}: la inexistencia de un componente en KNIME para AutoML que contenga un pre-procesado con las tareas automatizadas e incorpore la optimización de hiperparámetros. \\
En aras de resolver la problemática planteada se propone una nueva versión del componente \textit{AutoML (Componente AutoML Clasificación (pre-procesado)} \citep{Carrazana2022}, dado que este permite modificaciones, a diferencia del resto. Por tanto, se determina como \textbf{objetivo general} desarrollar una nueva versión del componente KNIME que permita automatizar tareas en el pre-procesado y la selección de hiperparámetros. Este objetivo general se desglosa en los siguientes \textbf{objetivos específicos y tareas}:

\begin{enumerate}
	\item Analizar el estado del arte de las principales técnicas de Aprendizaje Automático Automatizado para el pre-procesado y la optimización de hiperparámetros.
	\begin{enumerate}
		\item Describir las características del proceso KDD y Minería de Datos. 
		\item Caracterizar las técnicas de Aprendizaje Automático y las principales tareas de Automatización del Aprendizaje Automático.
		\item Asimilar componente AutoML Clasificación (pre-procesado).
	\end{enumerate}
	\item Desarrollar flujos de AutoML en KNIME en tareas de clasificación.
	\begin{enumerate}
		\item Desarrollar subcomponentes en KNIME para la automatización de actividades en el pre-procesado. (discretización, normalización, tratamiento de valores faltantes y de valores únicos).
		\item Desarrollar un componente KNIME para la optimización de hiperparámetros.
	\end{enumerate}
	\item Validar subcomponentes de actividades del pre-procesado de datos y componente AutoML Clasificación (Optimización de Hiperparámetros).
	\begin{enumerate}
		\item Desarrollar los casos de pruebas que permitan validar los subcomponentes de pre-procesado de datos propuestos. 
		\item Desarrollar los casos de pruebas que permitan validar el componente AutoML Clasificación (Optimización de Hiperparámetros).
		\item Evaluar los resultados arrojados por los casos de prueba.
	\end{enumerate} 
	\item Validar integración al componente AutoML Clasificación (pre-procesado)
	\begin{enumerate}
		\item Diseñar y ejecutar los casos de pruebas que permitan validar la integración de los subcomponentes de pre-procesado de datos y el componente AutoML Clasificación (Optimización de Hiperparámetros) al componente AutoML Clasificación (pre-procesado).
		\item Evaluar los resultados arrojados por los casos de prueba.
	\end{enumerate}
\end{enumerate}

%El \textbf{valor práctico} de este proyecto consiste en las modificaciones desarrolladas al componente KNIME de AutoML para el pre-procesado en tareas de clasificación, capaz de automatizar la optimización de hiperparámetros y ejecutar flujos de pre-procesado para diferentes algoritmos en dicha tarea. \\
En cuanto a la estructuración, este trabajo está dividido en tres capítulos:
\begin{itemize}
	\item \textbf{Capítulo 1: Aprendizaje Automático y AutoML}, se presenta el estudio realizado sobre las temáticas que aborda el trabajo, se muestran los conceptos fundamentales relacionados con el Aprendizaje Automático, la Minería de Datos y AutoML; así como la descripción del componente KNIME de AutoML para pre-procesado.
	\item \textbf{Capítulo 2: Propuesta de modificación al componente KNIME de AutoML para pre-procesado}, se presenta y expone el diseño e implementación de la modificación al componente KNIME para tareas de AutoML en pre-procesado y optimización de hiperparámetros.
		\item \textbf{Capítulo 3: Integración y validación de soluciones propuestas al componente de AutoML}, se muestran, comparan y analizan los resultados obtenidos de los algoritmos para diferentes configuraciones.
\end{itemize}




%Las figuras deben referenciarse en el texto, así como lo muestra este ejemplo, ver Figura \ref{fig:figCUJAE}.

%\begin{figure}[H] %la opción H indica al compilador LaTeX que posicione la figura lo más cerca posible de este lugar.
%\centering
 % \includegraphics[width=0.5\linewidth]{figuras/membrete-cujae-centrado.png}
 % \caption{El título de la figura debe estar acorde con su contenido.}
 % \label{fig:figCUJAE} %incluir el label permite referenciarla en cualquier parte del documento.
%\end{figure}

%Se deben utilizar siempre los mismos términos para referirse a los mismos conceptos y no olvidar de definir los términos que son claves en el campo de acción, o sea, la propuesta de la tesis.\\

%Describir la Situación problemática. Problema. Objetivo general. Objetivos específicos. Tareas. Beneficios. Breve resumen del contenido de la tesis.	
	
	\fancyhf{}
	\lhead[\thepage]{\leftmark}
	\rhead[\rightmark]{\thepage}
	\renewcommand{\chaptermark}[1]{\markboth{\chaptername \, \thechapter. #1}{}}
	
	\chapter{Aprendizaje Automático y AutoML}\label{chap:1}

En este primer capítulo, se aborda el marco teórico de la tesis, el cual se enfoca en diferentes temas clave dentro del campo de la inteligencia artificial y la minería de datos. En particular, se discute el aprendizaje automático, el descubrimiento de conocimiento en bases de datos, la minería de datos y la clasificación. Además, se introduce el concepto de AutoML, como herramienta para automatizar el proceso de pre-procesado y optimización de hiperparámetros en la implementación de técnicas de aprendizaje automático. Por último, se destaca la importancia de la plataforma KNIME como una herramienta útil para la implementación de técnicas de AutoML y análisis de datos.

\section{Proceso de descubrimiento de conocimiento en bases de datos}\label{kdd}
La Extracción de Conocimiento en Bases de Datos (\textit{Knwodlege Discovery from Databases}, o \textit{KDD} por sus siglas en inglés) se define como: “el proceso no trivial de identificar patrones válidos, novedosos, potencialmente útiles y, en última instancia, comprensibles a partir de los datos” \citep{orallo2004}. Este proceso está compuesto por una serie de etapas o fases, descritas a continuación:
\begin{itemize}
	\item Integración y recopilación: Es donde se decide qué datos serán utilizados en el proceso de KDD. Esto implica la selección de fuentes de datos relevantes y la adquisición de los conjuntos de datos necesarios.  Como parte del desarrollo de esta fase, es necesario diseñar o conocer el modo en que se van a organizar e integrar los datos; con el fin de eliminar redundancias e inconsistencias.
	\item Selección, limpieza y transformación: Se seleccionan los datos más relevantes y que aporten mejor información, garantizando que tengan la mejor calidad posible, logrando obtener la vista minable con los datos listos para la aplicación del algoritmo.
	\item Algoritmos de Minería de datos: En esta etapa central del proceso, se aplican algoritmos de minería de datos para identificar patrones, tendencias o estructuras en los datos. Esto puede incluir la clasificación, la segmentación, la regresión y otras técnicas de análisis.
	\item Evaluación e Interpretación: El objetivo de esta etapa es medir la calidad de los modelos obtenidos, utilizando diferentes métricas de calidad, las cuales dependen de las técnicas de minería de datos que se utilicen. La interpretación de los resultados se apoya en el uso de técnicas de visualización y de representación, con el fin de comprender mejor el conocimiento aportado. 
	\item Difusión y uso: En esta etapa, se integra el conocimiento obtenido de la comprensión del negocio, con el conocimiento de los modelos de minería de datos usado en la toma de decisiones de los especialistas. La monitorización de los patrones debe realizarse, pues en ocasiones resulta necesaria la reevaluación del modelo, su reentrenamiento o incluso su reconstrucción total.
\end{itemize}

En el contexto de la creciente acumulación de datos en instituciones que han digitalizado sus registros históricos en bases de datos, la extracción de información valiosa a través de patrones ocultos se convierte en un desafío crucial. La magnitud de esta información a menudo supera las capacidades de análisis de los expertos, lo que hace imperativo recurrir a técnicas automatizadas. Por lo tanto, la Minería de Datos emerge como una necesidad vital dentro del proceso KDD, ya que permite desentrañar conocimiento significativo y no evidente en estos vastos conjuntos de datos.

\subsection{Minería de Datos}
Acorde a \citep{orallo2004}, la minería de datos es definida como el proceso de extraer conocimiento útil y comprensible, previamente desconocido, desde grandes cantidades de datos almacenados en distintos formatos. \\
El conocimiento extraído se puede presentar en forma de relaciones, patrones o reglas inferidas de los datos, o en forma de descripción un poco más concisa. Estos constituyen un modelo de datos analizados. Estos modelos, o tareas, se categorizan en predictivas y descriptivas \citep{orallo2004}. \\
En las tareas predictivas, los ejemplos están etiquetados y se emplean para estimar valores futuros o desconocidos de variables de interés. En este entorno se encuentra el aprendizaje supervisado. En cambio, las tareas descriptivas son empleadas en el descubrimiento de propiedades de los datos examinados donde los ejemplos no se encuentran etiquetados. Aquí se pone de manifiesto el aprendizaje no supervisado. En \citep{orallo2004} se describen las tareas de minería de datos de la siguiente manera:
\begin{itemize}
	\item Clasificación: La clasificación se encarga de examinar las características de un registro u objeto, y de esta forma asignarle una clase predefinida. Estas clases son valores discretos. Para ello, se tiene que construir un modelo a partir de datos previamente clasificados. Como variantes a la clasificación, existe el aprendizaje de “rankings”, aprendizaje de preferencias y el aprendizaje de probabilidad, entre otros. 
	\item Regresión: A diferencia de la clasificación, el valor a predecir es numérico. Consiste en aprender una función real que calcula un valor para un atributo real. Su objetivo es minimizar el error entre el valor predicho y el valor real.
	\item  Correlaciones: Son empleadas para examinar el grado de similitud de los valores de dos variables numéricas. Se basa en el cálculo de correlación de variables numéricas usando la estadística. Este método trata de determinar si el comportamiento de dos variables numéricas está relacionado.
	\item Reglas de asociación: Son situaciones o características que ocurren en un mismo instante de tiempo. Pueden ser relaciones causales o casuales. Representan patrones de comportamiento entre los datos en función de la aparición conjunta de valores de dos o más atributos. Las medidas habituales propuestas en \citep{Agrawal1519}	para establecer la idoneidad y el interés de una regla de asociación son la confianza y el soporte.
	\item	Agrupamiento (Clustering): Para realizar esta tarea se parte de datos sin clasificar, teniendo como objetivo segmentar un grupo de datos diversos en subgrupos. Los miembros de cada grupo (clúster, por su definición en inglés) deben tener mucho en común entre sí y, a su vez, diferenciarse del resto de elementos de otros grupos. Dado que la clasificación de estos grupos no se conoce previamente, es el minero el encargado de darles un significado.
\end{itemize}

La minería de datos, como etapa inicial en la exploración y extracción de conocimiento a partir de grandes conjuntos de datos, sienta las bases para el aprendizaje automático, una disciplina de la inteligencia artificial que se centra en el desarrollo de algoritmos y modelos capaces de aprender patrones y realizar tareas de toma de decisiones basadas en datos.

\section{Aprendizaje Automático}
Uno de los campos más destacados dentro de la Inteligencia Artificial es el Machine Learning (Aprendizaje Automático), que es un enfoque que utiliza algoritmos y modelos matemáticos para permitir que los sistemas aprendan de los datos y realicen tareas específicas sin ser programados explícitamente.
Se define el Aprendizaje Automático como un conjunto de métodos que pueden detectar automáticamente patrones en los datos y luego, usar los patrones descubiertos para predecir datos futuros, o para realizar otros tipos de toma de decisiones bajo incertidumbre \citep{murphy2012machine}. Es decir, es el proceso en el que las computadoras descubren cómo hacer cosas sin estar específicamente programadas para hacerlo \citep{Praba2021}. Por lo tanto, el objetivo principal del aprendizaje automático es estudiar, diseñar y mejorar modelos matemáticos que se pueden entrenar (una vez o continuamente) con datos relacionados con el contexto (proporcionados por un entorno genérico), para inferir el futuro y tomar decisiones sin completo conocimiento de todos los elementos que influyen (factores externos) \citep{bonaccorso2017machine}. \\
Existen varios tipos de aprendizaje en Machine Learning, cada uno con sus propias técnicas y enfoques. A continuación, se presenta una breve descripción de algunos de los tipos de aprendizaje más comunes:
\begin{itemize}
	\item Aprendizaje supervisado: se refiere a cualquier proceso de aprendizaje automático que aprende una función de un tipo de entrada a un tipo de salida, utilizando datos que comprenden ejemplos que tienen valores de entrada y salida. Dos ejemplos típicos de aprendizaje supervisado son el aprendizaje de clasificación y la regresión \citep{sammut2011encyclopedia}. 
	\item Aprendizaje no supervisado: se refiere a cualquier proceso de aprendizaje automático que busca aprender la estructura en ausencia de un resultado identificado o retroalimentación. Tres ejemplos típicos de aprendizaje no supervisado son agrupamiento, reglas de asociación y mapas de autoorganización \citep{sammut2011encyclopedia}. 
	\item Aprendizaje por refuerzo: describe una gran clase de problemas de aprendizaje, característicos de los agentes autónomos que interactúan en un entorno: problemas de toma de decisiones secuenciales con recompensa retrasada. Los algoritmos de aprendizaje por refuerzo buscan aprender una política (mapeo de estados a acciones) que maximice la recompensa recibida a lo largo del tiempo. A diferencia de los problemas de aprendizaje supervisado, en los problemas de aprendizaje por refuerzo no hay etiquetas de ejemplo de comportamiento correcto e incorrecto. Sin embargo, a diferencia de los problemas de aprendizaje no supervisados, se puede percibir una señal de recompensa \citep{sammut2011encyclopedia}.
	\item Aprendizaje semisupervisado: es una clase de técnicas de aprendizaje automático que hacen uso de ejemplos etiquetados y no etiquetados para aprender un modelo. Los ejemplos etiquetados se usan para aprender modelos de clase y los ejemplos no etiquetados se usan para refinar los límites entre clases \citep{Han2011}.
\end{itemize}

A medida que la cantidad de datos disponibles continúa creciendo exponencialmente, y la complejidad de los modelos de Machine Learning aumenta, surge una necesidad cada vez mayor de contar con herramientas y técnicas que permitan a los usuarios automatizar el proceso de construcción de modelos. Es aquí donde entra en juego el Aprendizaje Automático Automatizado.

\begin{comment}
En particular, una de las técnicas centrales del aprendizaje supervisado es la clasificación, mencionada anteriormente como una de las tareas de la minería de datos, que permite a los algoritmos asignar etiquetas a datos y reconocer patrones subyacentes. Esta habilidad de clasificar datos en categorías predefinidas se traduce en una amplia gama de aplicaciones, lo que demuestra su versatilidad en la resolución de problemas del mundo real.
\end{comment}

%Texto... Es una buena práctica culminar (o iniciar) cada epígrafe con una referencia al que sigue (o precede) para dar fluidez a su lectura y no se vea como un \emph{copia y pega} sin ninguna relación.
         
\section{\textit{AutoML}}
El campo del Aprendizaje Automático Automatizado (AutoML), tiene como objetivo tomar decisiones de una manera automatizada, objetiva y basada en datos: el usuario simplemente proporciona datos y el sistema AutoML determina automáticamente el enfoque que funciona mejor para esta aplicación en particular \citep{hutter2019automated}. \\
AutoML (Automated Machine Learning) es una técnica que tiene como objetivo automatizar todo o parte del proceso de Machine Learning, incluyendo la selección de algoritmos, la optimización de hiperparámetros, la selección de características y la evaluación del rendimiento del modelo \citep{he2021automl}, \citep{tuggener2019automated}. La relación entre AutoML y Machine Learning es que AutoML es una técnica que se utiliza para automatizar el proceso de Machine Learning, por tanto se utiliza para automatizar todo o parte del proceso de selección del mejor modelo de Machine Learning, para un conjunto de datos dado. La automatización del proceso de Machine Learning proporciona una solución eficiente y escalable para el análisis de grandes conjuntos de datos, lo que puede resultar en un ahorro significativo de tiempo y recursos para los profesionales de ciencia de datos e investigadores. \\ 
Tras un análisis del estado del arte acerca del proceso de AutoML \citep{tuggener2019automated}, \citep{waring2020automated}, \citep{hutter2019automated}, \citep{he2021automl}, se pueden presentar como tareas principales las siguientes:
\begin{itemize}
	\item Selección de características: Esta tarea consiste en identificar las variables más relevantes para el problema de aprendizaje automático. 
	\item Pre-procesamiento de datos: La calidad de los datos de entrada es un factor crítico en el rendimiento de los modelos de aprendizaje automático. Las técnicas de preprocesamiento de datos se utilizan para limpiar, normalizar y transformar los datos de entrada en un formato que sea adecuado para el modelo. Existen varias técnicas efectivas de preprocesamiento de datos, incluyendo la eliminación de valores atípicos, la imputación de valores faltantes, la normalización y discretización de datos.
	\item Selección de modelo: Se basa en identificar el modelo de aprendizaje automático que mejor se ajusta al problema dado.
	\item Ajuste de hiperparámetros: Los modelos de aprendizaje automático tienen varios parámetros que afectan su rendimiento, y encontrar los valores óptimos de estos parámetros es una tarea importante para mejorar el rendimiento del modelo. El estado del arte ha demostrado que existen varias técnicas para el ajuste de hiperparámetros, incluyendo la búsqueda aleatoria \citep{zoller2021benchmark} y la optimización bayesiana \citep{he2021automl}, \citep{hutter2019automated}.
	\item Evaluación del modelo: La evaluación del modelo es una tarea crítica para medir el rendimiento del modelo en datos de prueba para determinar su capacidad para generalizar. Existen varias técnicas para la evaluación del modelo, incluyendo la validación cruzada y la evaluación de curvas de aprendizaje.
	\item Interpretación del modelo: Consiste en analizar el modelo de aprendizaje automático para comprender cómo se toman las decisiones y qué variables son importantes para la predicción.
\end{itemize}

En el marco de la presente investigación, se aborda de manera exhaustiva el enfoque fundamental de AutoML, centrándonos en dos aspectos cruciales que se explorarán en los epígrafes posteriores. En particular, se analizarán con detalle el pre-procesado de datos, que se enfoca en la preparación y limpieza de conjuntos de datos, y la optimización de hiperparámetros (HPO), que se concentra en la búsqueda de configuraciones óptimas para los modelos. Estos temas esenciales constituyen la base de esta investigación y servirán como pilares para la posterior discusión y análisis de resultados en este estudio. 

\subsection{Pre-procesado} \label{epig:preprocesado}
Una tarea importante en el proceso de AutoML es el pre-procesamiento de datos, siendo el conjunto de técnicas utilizadas para preparar los datos de entrada antes de alimentarlos a un modelo de aprendizaje automático. Esta etapa es la equivalente a la fase de selección, limpieza y transformación del proceso de KDD, descrita brevemente en la sección \ref{kdd}. \\
El pre-procesamiento de datos ayuda a mejorar la calidad de los datos de entrada y puede mejorar el rendimiento del modelo. La automatización del pre-procesamiento de datos a través de herramientas de AutoML, puede mejorar la eficiencia del proceso y ayudar al personal especializado, o sin mucha experiencia en el campo, a trabajar de manera más efectiva. Entre las técnicas de pre-procesado de datos se encuentran la discretización, la normalización, el tratamiento de valores faltantes y de atributos de tipo \textit{string}.

\subsubsection*{Discretización} \label{sub-epigrafe-disc}
Este procedimiento transforma datos cuantitativos en datos cualitativos, es decir, atributos numéricos en atributos discretos o nominales con un número finito de intervalos, obteniendo una partición no superpuesta de un dominio continuo. Luego se establece una asociación entre cada intervalo con un valor numérico discreto. Una vez realizada la discretización, los datos pueden ser tratados como datos nominales durante cualquier proceso de minería de datos \citep{garcia2015data}, \citep{garcia2012survey}. \\
Es necesario realizar la discretización de variables porque, entre varios factores, muchos de los algoritmos de Aprendizaje Automático requieren el uso de valores nominales solamente, como es el caso de ID3, Redes Bayesianas y Apriori. Otras ventajas derivadas de la discretización son la reducción y simplificación de datos, agilizando el aprendizaje y arrojando resultados más precisos, compactos y breves; y se reduce el posible ruido presente en los datos \citep{garcia2012survey}. No obstante, cualquier proceso de discretización conlleva generalmente una pérdida de información, siendo la minimización de esta pérdida el objetivo principal de un discretizador. \\
La elección del número de bins o intervalos para la discretización de datos es un proceso importante en el análisis de datos cuantitativos. El número de bins determina cómo se agrupan los valores continuos en categorías o intervalos discretos y puede influir en la interpretación de los resultados. Normalmente, este número es a elección del usuario, a pesar de que no existe una regla única para determinar el número de bins. En la literatura se han encontrado otros métodos, como:
\begin{itemize}
	
	\item Regla de Sturges \citep{coria2013mineria}: Propuesta por Herbert Sturges en 1926 \citep{sturges1926choice}, es una regla práctica para la selección del número de clases que se deben considerar al elaborar un histograma. Este número (\textit{k}) viene dado por la fórmula \ref{sturges}, donde \textit{n} es el tamaño de la muestra.
	\begin{equation} \label{sturges}
		k = 1 + \log_{2} n
	\end{equation}
	
	\item Validación cruzada \citep{torgo1997search}: La idea básica es probar diferentes configuraciones de parámetros y elegir la que proporcione la mejor precisión estimada. Esta mejor configuración encontrada se utilizará luego en el algoritmo de aprendizaje en la evaluación real, utilizando un conjunto de pruebas independiente. En cuanto al componente de búsqueda, se utiliza el escalador de colinas (hill-climbing) junto con un parámetro de anticipación configurable para minimizar el conocido problema de los mínimos locales. Para la estrategia de validación se emplea la validación cruzada.
	
	\item Tamaño de la muestra \citep{dougherty1995supervised}: Se calcula k teniendo en cuenta el número de valores distintos observados en el conjunto de entrenamiento, utilizando
	\begin{equation}
		k = max(1, 2 * \log l )
	\end{equation}
	donde \textit{l} es el número de valores distintos observados para cada atributo.
	
\end{itemize}

La identificación del mejor discretizador para cada situación es una tarea muy difícil de llevar a cabo. A pesar de la riqueza de la literatura, y aparte de la ausencia de una categorización completa de los discretizadores usando una notación unificada, hay pocos intentos de compararlos empíricamente. Esto se debe a que la evaluación de resultados es un tema complejo y depende de la necesidad del usuario en una aplicación en particular; además, la evaluación se puede hacer de muchas maneras. Existen tres dimensiones importantes según \citep{liu2002discretization}: 
\begin{enumerate}
	\item El número total de intervalos: intuitivamente, cuantos menos puntos de corte, mejor será el resultado de la discretización.
	\item El número de inconsistencias causadas por la discretización: no debe ser mucho mayor que el número de inconsistencias de los datos originales antes de la discretización.
	\item Precisión predictiva: cómo la discretización ayuda a mejorar la precisión.
\end{enumerate}
En resumen, se necesitan al menos tres dimensiones: simplicidad, consistencia y precisión. Idealmente, el mejor resultado de discretización puede obtener la puntuación más alta en los tres departamentos. En realidad, puede no ser alcanzable o necesario. Existen diversos algoritmos y enfoques utilizados para realizar esta conversión:

\begin{itemize}
	
	\item Agrupamiento (binning): es el método más simple para discretizar un atributo de valor continuo mediante la creación de un número específico de grupos (bins). Los bins se pueden crear con el mismo ancho (\textit{equal-width}) o la misma frecuencia (\textit{equal-frequency}). Cada bin está asociado con un valor discreto distinto. En \textit{equal-width}, el rango continuo de una característica se divide uniformemente en intervalos que tienen un ancho igual, y cada intervalo representa un bin. En \textit{equal-frequency}, se coloca un número igual de valores continuos en cada bin \citep{liu2002discretization}, \citep{yang2009discretization}. 
	
	\item Basado en cuantiles (\textit{quantiles-based}): Un método simple y similar a los anteriores, donde se producen bins correspondientes a una lista de probabilidades dada. El elemento más pequeño corresponde a una probabilidad de 0 y el más grande a una probabilidad de 1. 
	
	\item CAIM (Clustering and Mutual Information): El objetivo es encontrar el número mínimo de intervalos discretos mientras se minimiza la pérdida de interdependencia de atributo de clase. El algoritmo utiliza información de interdependencia de atributo de clase como criterio para la discretización óptima \citep{kurgan2004caim}. 
	
\end{itemize}

Existen otros métodos, que según el estudio realizado en \citep{garcia2012survey} son los más empleados, aparte de los mencionados anteriormente, como MDLP \citep{FI1993}, ID3 \citep{quinlan1993c}, ChiMerge \citep{K1992}, 1R \citep{holte1993very}, D2 \citep{catlett1991changing}, y Chi2 \citep{liu1997feature}. Sin embargo, en este contexto de investigación, es importante destacar que estos no serán empleados para la discretización de los atributos. La decisión se basa en la disponibilidad de implementaciones en KNIME de los métodos de agrupamiento, el basado en cuantiles (quantiles-based) y el algoritmo CAIM. Esta selección se realiza en consonancia con la revisión de la literatura y la investigación existente en el campo.

\subsubsection*{Normalización}
Las variables del conjunto de datos pueden variar en términos de magnitud, rango y unidad. Esto puede afectar negativamente el rendimiento de algoritmos, como SVM y redes neuronales, que emplean, en su mayoría, datos numéricos. En aras de resolver este problema, se realiza la normalización. Esta consiste  en ajustar los valores de una variable para que estén dentro de un rango específico o sigan una distribución particular. Se ejecuta principalmente para garantizar que las diferencias en la escala de las variables no afecten negativamente el rendimiento de los algoritmos de aprendizaje automático; y para facilitar la interpretación de los datos.\\
Existen varias técnicas de normalización comunes, donde algunas de las más utilizadas son la normalización Min-Max, Z-Score y Escala Decimal \citep{garcia2012survey}. En la literatura se pueden encontrar otras técnicas, descritas a continuación:
\begin{itemize}
	\item Normalización Min-Max: Transforma los valores de una variable para que estén en un rango específico, generalmente entre 0 y 1. Desventaja: Es sensible a valores atípicos.
	\item Normalización Z-score (estandarización): Transforma los valores para que tengan una media de 0 y una desviación estándar de 1. Desventaja: No escala los valores a un rango específico.
	\item Normalización de Escala Decimal: Transforma los valores desplazando el punto decimal, usando una división de potencia de diez, de modo que el valor absoluto máximo sea siempre inferior a 1 después de la transformación. Desventaja: Puede llevar a una pérdida significativa de precisión en los valores pequeños. 
	\item Normalización sólida \citep{polatgil2022investigation}: Este método es más útil especialmente cuando hay un valor atípico en los datos porque utiliza el valor mediano en lugar de la media y el rango de cuartiles en lugar del rango de valores. Sin embargo, la desventaja es que esta reducción de influencia de valores atípicos puede llevar a una pérdida de sensibilidad a la variabilidad de los datos.
	\item Escalador de Máximo Absoluto (Max Abs): En este método cada característica de los datos se obtiene dividiendo su valor máximo absoluto \citep{polatgil2022investigation}. Desventaja: No escala los valores a un rango específico, lo que puede dificultar la interpretación.
	\item Normalización AMZD: Propuesta en \citep{patro2015normalization} y similar a Min-Max, transforma los valores de datos de manera que tengan una representación en un rango de 0 y 1, con la diferencia de que funciona a nivel individual, lo que significa que se aplica a cada elemento de datos por separado; es independiente del tamaño del conjunto de datos; y
	es independiente del número de dígitos en cada elemento de datos. Sin embargo,  presenta como desventaja que solo puede ser aplicada a datos enteros.
\end{itemize}

En el contexto de AutoML (Aprendizaje Automático Automatizado), donde se busca una automatización eficiente del proceso de construcción de modelos, es crucial identificar un normalizador que sea coherente con los algoritmos utilizados. Algunos algoritmos pueden ser más sensibles a la escala y la distribución de las variables que otros. Por lo tanto, en esta investigación se realizan experimentos para evaluar el rendimiento de los modelos con diferentes técnicas de normalización y se selecciona la que maximice la calidad de las predicciones. En el capítulo \ref{chap:2} se profundiza en esta tarea.

\subsubsection*{Tratamiento de valores faltantes} \label{secc:mv}
Los valores faltantes o perdidos son datos que no aparecen en la celda de la columna que les corresponde. Esto puede ocurrir por diversas razones, como errores en la recolección de datos, omisiones intencionales, fallos en la medición o simplemente porque no se disponía de información en el momento de la recopilación. Esto trae consigo varias dificultades: disminución de la eficacia, complicaciones en la gestión y análisis de datos; así como a resultados sesgados. Sin embargo, los datos incompletos son objeto de investigación debido a su influencia en la precisión de la clasificación. Normalmente, los valores perdidos pueden ser manejados de varios métodos \citep{garcia2015data}, \citep{ventevogel2020construction}:
\begin{itemize}
	\item Descartar los valores: Consiste en eliminar por completo los registros que contienen valores faltantes. Descartar completamente un atributo es un enfoque que se puede utilizar cuando se observan muchos valores perdidos para el mismo. Una regla general es que si faltan más del 60-70\% de los valores, es mejor eliminar el atributo \citep{ventevogel2020construction}. Sin embargo, este método puede resultar en la pérdida de información valiosa y reducir el tamaño de la muestra.
	\item Imputación mediante muestreo: Este enfoque es más adecuado para funciones no categóricas. Al aplicarlo, se utilizan procedimientos de máxima verosimilitud para estimar los parámetros de la porción completa de los datos. Utilizando estos parámetros, los valores perdidos se completan mediante muestreo de esta distribución estimada.
	\item Imputación múltiple: Implica reemplazar los valores faltantes por un valor constante, como la media, la mediana o la moda de la variable en cuestión. Este enfoque es útil cuando los valores faltantes son aleatorios y no se espera que sigan un patrón específico. Para datos faltantes más complejos, es posible utilizar modelos predictivos para estimar los valores perdidos. Esto puede incluir regresiones, árboles de decisión o métodos de Machine Learning más avanzados.
\end{itemize}
En el contexto de este análisis, nos enfocamos en este último método. Existe una amplia familia de métodos de imputación, desde técnicas de imputación simples como sustitución de medias, KNN, etc.; a aquellos que analizan las relaciones entre atributos tales como: basados en SVM, basados en clustering, regresiones logísticas, procedimientos de máxima verosimilitud e imputación múltiple. En esta investigación se emplean los métodos KNN y K-Means para la imputación de estos valores, dado que en KNIME se tiene la posibilidad de implementarlos y en la literatura se ha demostrado que son robustos para esta tarea \citep{tsai2022empirical}, \citep{batista2003analysis}, \citep{patil2010missing}, \citep{li2004towards}.\\
K-Vecinos Más Cercanos (K-Nearest Neighborst o KNN) es un algoritmo de aprendizaje supervisado, que se utiliza comúnmente para la imputación de valores faltantes en conjuntos de datos. La idea básica es encontrar las \textit{k} observaciones más cercanas a la observación con el valor faltante, y utilizar los valores de esas observaciones para estimar el valor faltante. \\
Por otra parte, K-Medias (K-Means) es un algoritmo de aprendizaje no supervisado que se utiliza para la agrupación de datos. Se puede usar como parte de un enfoque más amplio para la imputación, por ejemplo, para agrupar observaciones similares y luego imputar valores faltantes dentro de cada grupo utilizando técnicas específicas, como la imputación de la media o la mediana del grupo.

\subsubsection*{Tratamiento de atributos de alta cardinalidad}\label{alta-cardinalidad}
Los atributos categóricos, en contraste con los atributos numéricos, representan categorías o etiquetas. Se dividen en dos tipos principales: nominales, que representan categorías sin orden específico (como colores o países), y ordinales, que muestran un orden jerárquico (como niveles educativos). En muchos conjuntos de datos, es común encontrarse con atributos categóricos que presentan una alta cardinalidad. La cardinalidad de un atributo categórico esta definida por el número de valores distintos que un atributo puede tomar \citep{moeyersoms2015including}. Las variables categóricas de alta cardinalidad son variables para las cuales el número de niveles diferentes es grande en relación con el tamaño de la muestra de un conjunto de datos. \\
Estos atributos presentan desafíos significativos en el análisis de datos y requieren un manejo cuidadoso. Uno de los problemas clave que se enfrentan es la complejidad computacional, ya que una gran cantidad de categorías puede aumentar el tiempo de procesamiento y el consumo de memoria. Además, pueden llevar a problemas de sobreajuste en modelos de aprendizaje automático y dificultar la interpretación de resultados. Por ejemplo, en un país existen muchos importadores y exportadores diferentes. Al entrenar un modelo de aprendizaje automático, este tipo de datos no son útiles, ya que dificulta la generalización del modelo, pero con un método de codificación adecuado, se ha demostrado que estas características mejoran el rendimiento del modelo de clasificación \citep{hooi2022feature}, \citep{cerda2020encoding}. Para abordar estos problemas, existen varias estrategias, como el agrupamiento semántico, que consiste en identificar grupos significativos desde el punto de vista semántico \citep{cerda2018similarity}; y distintos métodos de codificación, como One-Hot, Target, Peso de Evidencia (WOE) y Radio Supervisado. \\
One-Hot Encoding transforma una característica categórica de \textit{q} categorías únicas en representaciones numéricas, construyendo \textit{q} atributos binarios, uno para cada categoría; \citep{avanzi2023machine}. En el ejemplo de la figura \ref{fig:one-hot-ejemplo}, los códigos ZIP de clientes individuales (C1, C2, etc.) se transforman utilizando este método, donde se crea una variable  para cada código ZIP. \\
En otras palabras, cada dimensión representa la ausencia o presencia de una categoría dada. En \citep{hooi2022feature} se realizan pruebas para comparar el desempeño de varias técnicas de codificación con respecto a algoritmos de Aprendizaje Automático, como SVM, árboles de decisión y redes neuronales, lo cual, con respecto al codificador One-Hot, se obtuvieron muy buenos resultados. \\
\begin{figure}[H]
	\centering
	\includegraphics[width=0.8\linewidth]{"figuras/capi 1/one-hot-ejemplo"}
	\caption{Ejemplo de One-Hot Encoding (también conocido como Dummy Encoding) para el atributo ZIP Code}
	\label{fig:one-hot-ejemplo}
\end{figure}
Este método, aunque eficaz para atributos de baja cardinalidad, presenta la desventaja de generar una alta dimensionalidad, lo que puede ocasionar desafíos computacionales y estadísticos. Para evitarlo, es necesario aplicar reducción de dimensionalidad en la matriz de características resultante. Un enfoque natural es utilizar el Análisis de Componentes Principales (PCA) \citep{mahmood2022accurate}, \citep{kasemtaweechok2021large} , donde la idea básica es encontrar un conjunto de transformaciones lineales de las variables originales que puedan describir la mayor parte de la varianza utilizando un número relativamente menor de variables \citep{garcia2015data}. Este enfoque permite la combinación de la esencia de los atributos originales para formar un nuevo subconjunto más pequeño de atributos. \\
La reducción de la cardinalidad comprende las transformaciones aplicadas para obtener una representación reducida de los datos originales. Un ejemplo de reducción de cardinalidad es la introducción de un valor \textit{otros}, que se asigna a los valores que representan menos de un cierto umbral en el atributo \citep{casas2019data}. Esto es especialmente útil cuando muchos valores ocurren muy pocas veces \citep{ventevogel2020construction}.

\subsection{Optimización de hiperparámetros} \label{epig:hpo}
La optimización de hiperparámetros es un componente esencial en el campo del aprendizaje automático. En el proceso de entrenar modelos, los hiperparámetros desempeñan un papel crítico al influir en el rendimiento y la capacidad de generalización de un algoritmo. La tarea de optimizar estos hiperparámetros implica encontrar la combinación óptima que maximice la eficiencia, precisión y rendimiento de un modelo en un conjunto de datos de prueba o validación  \citep{hastie2009elements}. La optimización de hiperparámetros automatizada (HPO) tiene varios casos de uso importantes  \citep{hutter2019automated}; puede
\begin{itemize}
	\item reducir el esfuerzo humano necesario para aplicar el aprendizaje automático. Particularmente importante en el contexto de AutoML,
	\item mejorar el rendimiento de los algoritmos de aprendizaje automático (adaptándolos al problema en cuestión), y
	\item mejorar la reproducibilidad y equidad de los estudios científicos.
\end{itemize}
Para ahorrar tiempo y mejorar la precisión de los modelos de aprendizaje automático, se combina la selección de algoritmos y la optimización de hiperparámetros en un solo proceso, denominado Selección de Algoritmos y Optimización de Hiperparámetros Combinados (\textit{CASH}, por sus siglas en inglés) \citep{tuggener2019automated}. CASH, en conjunción con la automatización del pre-procesado de los datos, integran el problema general del AutoML, reflejado en la figura \ref{fig:desglose-de-los-subproblemas-de-automl}. 

\begin{figure}[H]
	\centering
	\includegraphics[width=0.4\linewidth]{"figuras/capi 1/Desglose de los subproblemas de AutoML"}
	\caption{Desglose de los subproblemas de AutoML \citep{zoller2021benchmark}}
	\label{fig:desglose-de-los-subproblemas-de-automl}
\end{figure} 

Encontrar los valores óptimos para estos hiperparámetros es crucial para obtener resultados precisos y generalizables. Para lograr esto, existen diversas estrategias de optimización de hiperparámetros que permiten explorar eficientemente el espacio de búsqueda y encontrar las mejores configuraciones.  A continuación, se examinan algunas de estas estrategias:

\begin{itemize}
	\item Búsqueda aleatoria: Selecciona valores de forma aleatoria dentro de un rango definido. No garantiza encontrar los mejores valores posibles, y puede requerir numerosas iteraciones para encontrar un conjunto de hiperparámetros que proporcione un rendimiento óptimo \citep{geron2022hands},\citep{zoller2021benchmark}.
	\item Búsqueda voraz: Prueba todas las combinaciones posibles de valores de los hiperparámetros dentro de un rango definido. Presenta una alta demanda computacional, imposible de manejar a medida que escalan los sistemas y las bases de datos \citep{zoller2021benchmark}.
	\item Búsqueda en cuadrícula: Prueba cada combinación de valores de hiperparámetros y selecciona la combinación que haya dado el mejor rendimiento. Puede ser computacionalmente costoso si el espacio de búsqueda y el número de hiperparámetros es grande \citep{he2021automl}.
	\item Optimización Bayesiana: Construye modelos probabilísticos para representar la función objetivo, que se actualiza después de cada iteración. Maneja espacios de búsqueda de alta dimensionalidad, y no requiere tantas iteraciones, como la búsqueda en cuadrícula o la búsqueda aleatoria, para encontrar combinaciones de hiperparámetros de alto rendimiento \citep{hutter2019automated}, \citep{he2021automl}.
	\item Optimización basada en gradiente: Emplea la información del gradiente para optimizar los hiperparámetros de manera iterativa. Genera una gran carga computacional y presenta la limitación de caer en mínimos locales \citep{zoller2021benchmark}.
\end{itemize} 

Una vez identificadas las variables clave del modelo, es crucial evaluar su rendimiento de manera sólida y confiable. Encontrar los valores óptimos de hiperparámetros es crucial para lograr un modelo eficaz. En este contexto, las estrategias de validación desempeñan un papel esencial, ya que permiten evaluar y comparar diferentes combinaciones de hiperparámetros de manera sistemática y eficiente, garantizando que el modelo esté bien ajustado. En la literatura se encuentran estrategias de validación, como:

\begin{itemize}
	\item Validación cruzada: Implica dividir el conjunto de datos en múltiples subconjuntos, entrenando y evaluando el modelo en diferentes particiones para estimar su capacidad de generalización. Esta estrategia permite determinar cómo se comporta el modelo con diferentes combinaciones de datos de entrenamiento y prueba, lo que es especialmente importante cuando se trata de la selección de hiperparámetros y la evaluación de su impacto en el rendimiento \citep{hastie2009elements}, \citep{bishop2006pattern}.
	\item Validación holdout: En esta estrategia, se divide el conjunto de datos en tres partes: un conjunto de entrenamiento, un conjunto de validación y un conjunto de prueba. El conjunto de entrenamiento se utiliza para ajustar los hiperparámetros, el conjunto de validación se utiliza para seleccionar los mejores hiperparámetros y el conjunto de prueba se utiliza para evaluar el rendimiento final del modelo. Esta estrategia es útil cuando se dispone de suficientes datos y se desea tener una evaluación más realista del rendimiento del modelo \citep{bishop2006pattern}.
	\item Validación escalonada: Esta estrategia es similar a la validación holdout, pero se realiza en múltiples etapas. El conjunto de datos se divide en varios conjuntos, y se realiza un proceso iterativo en el que se entrena y se evalúa el modelo utilizando diferentes combinaciones de hiperparámetros. Esto permite obtener una evaluación más robusta del rendimiento del modelo y reducir el sesgo introducido por una única partición de los datos \citep{hastie2009elements}.
	\item Validación Bootstrap: En esta estrategia, se generan múltiples muestras de entrenamiento mediante muestreo con reemplazo de los datos originales. Luego, se entrena y evalúa el modelo para cada muestra y se promedian los resultados. Esto permite tener una estimación del rendimiento del modelo en el conjunto de datos original \citep{davison1997bootstrap}.
\end{itemize}
La elección de la estrategia adecuada depende del tamaño del conjunto de datos, la complejidad del modelo y la disponibilidad de recursos computacionales. Es importante seleccionar la estrategia que mejor se ajuste a las necesidades específicas del problema y los datos. \\
A pesar del incesante estudio vinculado al HPO, este sigue presentando en la actualidad diversos desafíos \citep{hutter2019automated}:
\begin{itemize}
	\item Costo computacional: la optimización de hiperparámetros puede ser muy costosa en términos de tiempo de cómputo y recursos de hardware, especialmente cuando se utiliza un espacio de hiperparámetros grande o se ejecutan muchas iteraciones de entrenamiento. Esto puede limitar la escalabilidad y la eficiencia de la HPO.
	\item Generalización del modelo: la optimización de hiperparámetros puede resultar en un modelo altamente ajustado, que no generaliza bien a nuevos datos. Se torna complejo cuando las bases de datos poseen múltiples tipos de datos.
	\item Complejidad del espacio de hiperparámetros: el espacio de hiperparámetros puede ser muy complejo y estar altamente interconectado, lo que dificulta la exploración y la selección de los hiperparámetros adecuados.
\end{itemize}

Entre las aplicaciones de HPO, se pueden encontrar dos ejemplos destacados en la literatura. En primer lugar, \citep{hernandeztecnicas} aplica HPO en SVM y bosques aleatorios para predecir enfermedades cardiovasculares. En segundo lugar, \citep{waring2020automated} aborda el desarrollo del problema de HPO en redes neuronales en un contexto de análisis de salud.

\begin{comment}
	
	
	\section{Clasificación}
	En el uso común, la palabra clasificación significa colocar las cosas en categorías, agruparlas de alguna manera útil. El ser humano generalmente hace esto porque las cosas en un grupo, llamado \textit{clase} en aprendizaje automático, comparten características comunes \citep{sammut2011encyclopedia}. \\
	En aprendizaje automático, la clasificación se utiliza para identificar a qué clase o categoría pertenece una determinada observación o registro, basándose en un conjunto de características o variables. En esta tarea, se utiliza un algoritmo para construir un modelo predictivo que asigne una etiqueta de clase a cada observación en función de sus características. Este modelo se entrena utilizando un conjunto de datos etiquetados previamente, y luego se aplica a nuevos datos para hacer predicciones \citep{sammut2011encyclopedia}. En las siguientes secciones se describen los algoritmos empleados en el componente AutoML Clasificación (pre-procesado).
	
	\subsubsection{Clasificación con Árboles de Decisión}
	La clasificación con árboles de decisión es un método popular en la minería de datos y en el aprendizaje automático supervisado, utilizado para predecir la clase o categoría de un objeto o registro. Los Árboles de Decisión son unos de los modelos más populares, su representación es de fácil entendimiento, incluso por personas no afines al área, pues su construcción en sencilla: las hojas toman los valores objetivos, mientras los atributos y sus posibles valores conforman los nodos y ramas respectivamente \citep{sammut2011encyclopedia}. \\
	Basados en árboles de decisión existen otros algoritmos de clasificación como: ID3, C4.5 y CART. Cada uno de ellos fue desarrollado como una versión mejorada del anterior, pero todos tienen una alta eficiencia y tiempos de ejecución reducidos, lo que los hace igualmente populares en la actualidad \citep{javed2022performance}. Por otra parte, el algoritmo Bosque Aleatorio (mejor conocido como Random Forest, por su traducción al inglés), se basa en la idea de combinar múltiples árboles de decisión para mejorar la precisión y la capacidad de generalización del modelo. Se comparan y analizan en \citep{gupta2017analysis}, donde se arrojan sus principales características:
	\begin{itemize}
		\item ID3 (Iterative Dichotomiser 3): Basa su funcionamiento en la entropía y la ganancia de información. El árbol se construye iterativamente de arriba hacia abajo, eligiendo en cada caso el atributo con mayor ganancia de información, hasta que la información ganada sea cero o se haya llegado a todas las hojas \citep{javed2022performance}, \citep{gupta2017analysis}. Maneja datos nominales y no tolera valores faltantes.
		\item C4.5 (Classification 4.5): Desarrollado con el objetivo de mejorar los defectos de ID3. Añade la poda, desestimando las ramas sin aportes, que reduce los errores al clasificar como resultado de un gran número de ramas en el modelo \citep{sammut2011encyclopedia}, \citep{javed2022performance}, \citep{gupta2017analysis}. Adicionalmente maneja valores faltantes y numéricos. 
		\item CART (Classification and Regression Trees): Genera un árbol binario siguiendo el mismo enfoque entrópico que ID3, pero empleando el coeficiente de Gini como criterio de selección \citep{javed2022performance}, \citep{gupta2017analysis}. Es capaz de manejar datos faltantes, al igual que datos numéricos y nominales.
		\item Random Forest: Es un algoritmo de aprendizaje automático en el que se construye un conjunto de árboles de decisión para realizar predicciones. Cada árbol se entrena con un subconjunto aleatorio de datos y características, y luego se combina su salida para obtener una predicción más robusta y precisa. Reduce el sobreajuste y mejora la generalización al promediar múltiples modelos \citep{gupta2017analysis}.
	\end{itemize}
	
	\subsubsection{Clasificación con Redes Neuronales}
	Las redes neuronales o redes neuronales artificiales, son algoritmos de aprendizaje basados en una vaga analogía de cómo funciona el cerebro humano. El aprendizaje se logra ajustando los pesos en las conexiones entre nodos, que son análogas a las sinapsis y las neuronas \citep{sammut2011encyclopedia}. \\
	Las primeras redes neuronales conocidas como pre-alimentadas, como la de la figura \ref{fig:red-neuronal-prealimentada}, se caracterizan por tener una arquitectura en la que cada capa de neuronas está conectada completamente con la capa siguiente, pero no con la capa anterior. Esto significa que la información fluye de forma unidireccional, sin retroalimentación \citep{abiodun2018state}. 
	\begin{figure}[H]
		\centering
		\includegraphics[width=0.7\linewidth]{figuras/capi 1/red-neuronal-prealimentada}
		\caption{Red Neuronal Pre-Alimentada \citep{abiodun2018state}}
		\label{fig:red-neuronal-prealimentada}
	\end{figure}
	Posteriormente, se desarrollaron las redes neuronales por retro-propagación, como la presente en la figura \ref{fig:red-neuronal-retropropagacion}, que comparan la salida obtenida por la red con la salida deseada, y ajustan los pesos de las conexiones entre las neuronas de la red para reducir el error de predicción. La retro-propagación se utiliza para calcular la contribución de cada peso en el error de la red, y así ajustarlos de manera adecuada \citep{abiodun2018state}.
	\begin{figure}[H]
		\centering
		\includegraphics[width=0.6\linewidth]{figuras/capi 1/red-neuronal-retropropagación}
		\caption{Red neuronal por retro-propagación \citep{abiodun2018state}}
		\label{fig:red-neuronal-retropropagacion}
	\end{figure}
	Gracias a su gran versatilidad se pueden aplicar en una amplia variedad de campos y disciplinas para resolver problemas complejos de aprendizaje automático, como es el procesamiento del lenguaje natural, reconocimiento de voz e imágenes \citep{abiodun2018state}. Debido a la complejidad de sus modelos puede ser difícil su entendimiento, por lo que preferiblemente se utilizan en el contexto del reconocimiento de patrones.
	
	\subsubsection{Clasificación con Support Vector Machine}
	Las Máquinas de Soporte Vectorial (Support Vector Machine o SVM, en inglés), son una clase de algoritmos lineales que se pueden emplear para la clasificación \citep{sammut2011encyclopedia}, cuyo objetivo es encontrar el hiperplano que mejor separa dos clases de datos en un espacio de alta dimensionalidad \citep{guenther2016support}. En la figura \ref{fig:ejemplos-de-posibles-hiperplanos-de-svm} se representa como el hiperplano H2 divide con mayor margen las clases que el hiperplano H1.
	\begin{figure}[H]
		\centering
		\includegraphics[width=0.4\linewidth]{"figuras/capi 1/Ejemplos de posibles hiperplanos de SVM"}
		\caption{Ejemplos de posibles hiperplanos de SVM \citep{sammut2011encyclopedia}}
		\label{fig:ejemplos-de-posibles-hiperplanos-de-svm}
	\end{figure}
	Uno de los desafíos de SVM en problemas de clasificación de múltiples clases es cómo manejar la predicción de estas. En este contexto, existen dos enfoques principales:
	\begin{itemize}
		\item Uno contra uno (One-vs-One): Durante la fase de predicción, cada clasificador binario vota por la clase que cree que es correcta y la clase con el mayor número de votos se selecciona como la clase final para el punto de datos dado. Este enfoque presenta la ventaja de que cada clasificador binario solo necesita ser entrenado en un subconjunto de los datos, lo que puede ser útil cuando hay grandes conjuntos de datos. Es más resistente a desequilibrios en la distribución de clases que otros enfoques de SVM \citep{guenther2016support}.
		\item Uno contra todos (One-vs-All): Entrena un clasificador binario para cada clase posible, donde el conjunto de datos de una clase se considera positivo y los datos de las otras clases se consideran negativos. Durante la fase de predicción, cada clasificador binario vota por su clase correspondiente y la clase con la mayor puntuación se selecciona como la clase final para el punto de datos dado. Fácil de implementar y puede funcionar bien en conjuntos de datos pequeños, pero puede ser menos preciso en conjuntos de datos grandes y complejos \citep{guenther2016support}.
	\end{itemize}
	Agregar que existen otros dos enfoques: Clasificación jerárquica (Hierarchical classification) y Clasificación por parejas (Pairwise classification). Cada uno de estos tiene sus propias ventajas y desventajas, y la elección de uno de ellos dependerá del tipo de datos y del problema que se esté tratando de resolver. Es importante mencionar que la clasificación es solo una de las muchas técnicas utilizadas en el Aprendizaje  Automático. 
	
\end{comment}

\section{Herramienta de minería de datos KNIME}

La herramienta de datos KNIME (\textit{Konstanz Information Miner}, por sus siglas en inglés), es una plataforma de minería de datos de código abierto, disponible para varias plataformas y sistemas operativos, que permite el desarrollo de modelos en un entorno visual. Esta herramienta tiene como objetivo desarrollar procesos de KDD a través de un entorno visual. Se le considera una herramienta gráfica, ya que permite construir flujos de trabajo \citep{knime2023}. \\
Los flujos se componen de flechas y nodos que se pueden combinar entre sí. Los nodos contienen funcionalidades tales como: algoritmos de minería de datos, formas de conexión a los datos almacenados, pre-procesamiento de datos, reportes, entre otros. Las flechas indican el orden de ejecución y el flujo de la información. En la figura \ref{fig:ejemploworkflow}, se muestra un ejemplo de un flujo en KNIME para cargar y filtrar datos de una tabla, y posteriormente guardar los resultados en un fichero .csv. 
\begin{figure}[H]
	\centering
	\includegraphics[width=0.9\linewidth]{figuras/capi 1/ejemplo_workflow}
	\caption{Ejemplo de flujo de trabajo en la herramienta KNIME.}
	\label{fig:ejemploworkflow}
\end{figure}
Los metanodos son un tipo de nodo que contiene un subflujo, que pueden contener varios nodos y metanodos \citep{berthold2009knime}, comportándose como un flujo regular. Generalmente se emplea para organizar grandes flujos en varios subflujos que agrupen pequeñas metas, ganando así en claridad. Los componentes en KNIME pueden conformarse por flujos y metanodos, incluso por otros componentes. Estos, a diferencia de los metanodos, poseen la capacidad de configurar parámetros. \\
La herramienta KNIME puede ser extendida a través de plugins, la mayoría son nuevos nodos, aunque las extensiones pueden ser a cualquier parte de la arquitectura. La extensibilidad de la herramienta es de forma sencilla, ya que está basada en la Plataforma de Cliente Enriquecido de Eclipse (\textit{Eclipse RCP}, por sus siglas en inglés) \citep{berthold2009knime}. Gracias a esto, la adición de nuevos plugins a KNIME se torna menos compleja para el desarrollador. \\
KNIME se diseñó en base a tres principios: modularidad, extensibilidad y ambiente de trabajo interactivo. A continuación, se describen estos principios \citep{Lisandra2012}:

\begin{itemize}
	\item Modularidad: Plantea que no deben existir dependencias entre las unidades contenedoras de datos o de procesamiento. Además, se pueden implementar algoritmos de manera independiente. De igual forma, al no tener tipos de datos predefinidos, se pueden definir nuevos tipos de datos, con sus características y especificaciones propias. Estos pueden declararse compatibles con otros existentes.
	\item	Extensibilidad de forma sencilla: Permite adicionar nuevas unidades de procesamiento, visualización y tratamiento de datos, teniendo en cuenta que esto debe ser una tarea fácil de realizar.
	\item	Ambiente de trabajo visual e interactivo: Los flujos de trabajo deben ser fáciles e interactivos para el usuario. Por tal motivo, se harán arrastrando los elementos al área de trabajo.
\end{itemize}

El AutoML en KNIME puede ayudar a simplificar y agilizar el flujo de trabajo de aprendizaje automático, facilitando a los usuarios el desarrollo de modelos predictivos sin necesidad de una intervención manual extensa. Dado que se realiza la modificación a un componente en esta herramienta, se continúa el desarrollo en la misma.

\section{\textit{AutoML} en KNIME}
AutoML en KNIME se refiere a la capacidad de la plataforma para automatizar el proceso de modelado de aprendizaje automático. Esto significa que los usuarios pueden cargar datos y permitir que la plataforma seleccione y optimice automáticamente el modelo que mejor se ajuste a los datos. Con ese objetivo se han desarrollado extensiones y componentes, que permiten a los usuarios de KNIME automatizar tareas de aprendizaje automático, lo que ahorra tiempo y esfuerzo, especialmente para aquellos que pueden no ser expertos en Machine Learning. Sin embargo, al igual que con cualquier herramienta de AutoML, también pueden presentar desafíos, como la selección de algoritmos y la interpretación de modelos. Por lo tanto, es importante comprender las capacidades y limitaciones de las extensiones de AutoML en KNIME y cómo se integran en su flujo de trabajo de análisis de datos antes de su implementación. En las siguientes secciones se analizan de forma breve estas implementaciones.

\subsection{Extensión H2O.ai}
H2O es una plataforma de aprendizaje automático distribuida de código abierto diseñada para escalar a conjuntos de datos muy grandes, con API en R, Python, Java y Scala. H2O AutoML \citep{ledell2020h2o} es un algoritmo automatizado de aprendizaje automático incluido en el marco H2O que es fácil de usar y produce modelos de alta calidad que son adecuados para su implementación en un entorno empresarial. H2O AutoML admite el entrenamiento supervisado de modelos de regresión, clasificación binaria y clasificación multiclase en conjuntos de datos tabulares.\\
H2O.ai, la empresa detrás de H2O AutoML, ha desarrollado una integración con KNIME. Esta permite a los usuarios de la plataforma aprovechar las capacidades de H2O para construir modelos de aprendizaje automático en el entorno de KNIME, lo que facilita la creación de flujos de trabajo de análisis de datos completos, que incluyen tareas de preparación de datos y modelado predictivo. Dentro de la extensión H2O de KNIME, se halla el nodo H2O AutoML, que automatiza el proceso de aprendizaje automático. Este nodo lleva a cabo tareas como el entrenamiento y ajuste automático de diversos modelos, tales como la Máquina de Aumento de Gradiente, el Modelo Lineal Generalizado, las Redes Neuronales Profundas y el Bosque Aleatorio, todo ello dentro de un intervalo de tiempo predeterminado por el usuario. \\
H2O AutoML selecciona el modelo principal y, como parte del proceso de aprendizaje, optimiza automáticamente los hiperparámetros mediante una búsqueda de cuadrícula aleatoria. Además, implementa una validación cruzada con un valor predeterminado de 5 particiones como estrategia de validación. Cabe destacar que este nodo no contiene opciones de pre-procesado de los datos ni permite la visualización ni la modificación de su contenido, únicamente se puede acceder a la interfaz de configuración. Por otra parte, es necesaria la intervención humana para la implementación de los flujos.

\subsection{Componente AutoML KNIME}
Este componente de KNIME, incorporado en el 2021, tiene la capacidad de entrenar automáticamente modelos de aprendizaje automático supervisados para clasificación binaria y multiclase. Automatiza el ciclo de aprendizaje automático, abarcando desde la preparación de datos, la optimización de parámetros con validación cruzada, la puntuación, la evaluación y hasta la selección de modelos. Además, este componente captura de manera integral todo el proceso y genera un flujo de trabajo de implementación mediante la extensión integrada de KNIME. Su funcionamiento se puede resumir de la siguiente manera:

\begin{itemize}
	\item Preparación de datos: Antes de entrenar los modelos, se inicia la fase de preparación de datos. Durante esta etapa, se realiza una limpieza de los datos que implica reemplazar los valores faltantes en las columnas categóricas por la moda y en las columnas numéricas por la media. Existe la opción de aplicar codificación One-Hot a los datos categóricos y de eliminar columnas con un alto número de valores únicos, según un parámetro definido por el usuario. Además, todas las características numéricas se convierten en \textit{double} y se normalizan utilizando la normalización Z-Score. La partición de datos en conjuntos de entrenamiento y prueba se lleva a cabo automáticamente mediante muestreo estratificado en la clase objetivo.
	\item Entrenamiento de modelos: La etapa de entrenamiento de modelos implica la capacitación de diversos tipos de modelos, como Naive Bayes, Regresión Logística, Redes Neuronales, Árboles de Decisión, Bosques Aleatorios y otros. Cada modelo tiene parámetros ajustables que pueden ser optimizados utilizando validación cruzada y una métrica de evaluación personalizada en los datos de entrenamiento. 
	\item Selección del mejor modelo: Una vez que se ha completado el entrenamiento de los modelos, se procede a la selección del mejor. Todos los modelos entrenados se aplican al conjunto de pruebas y se obtienen predicciones. Estas predicciones se comparan con los valores reales, lo que permite calcular diversas métricas de rendimiento. El mejor modelo se selecciona basándose en una métrica de rendimiento especificada por el usuario.
\end{itemize}

Dentro de su configuración, se incluye la selección de los modelos a utilizar, permite seleccionar la partición de entrenamiento y la estrategia de validación, que tiene un valor predeterminado de 5. Sin embargo, debido a su formato caja negra, no se puede visualizar ni modificar los flujos correspondientes a las tareas de AutoML.

\subsection{Componente KNIME de AutoML Clasificación (pre-procesado)}
En \citep{Carrazana2022} se desarrolla un componente KNIME de AutoML para el pre-procesado en tareas de clasificación. Este, a partir de un conjunto de datos y una columna objetivo, ejecuta diferentes flujos de pre-procesado, en aras de cumplir con los requisitos de los diferentes algoritmos de clasificación, siendo capaz de entrenarlos y probarlos, para posteriormente puntuar y graficar los resultados. Este componente está enfocado en el pre-procesado de datos, donde se desarrollaron subcomponentes orientados a la realización de las tareas de pre-procesamiento de datos numéricos, \textit{string}, valores faltantes y el ajuste de tipos de columna. Los algoritmos de clasificación implementados son ID3, C4.5, CART, Redes Neuronales por Retropropagación (RProp), Redes Neuronales Probabilísticas (PNN) y Máquina de Soporte Vectorial (SVM). Cada uno de estos requiere un tipo de pre-procesado diferente, acorde a los tipos de datos con los que trabaja. Este componente consta de tres etapas clave:

\begin{itemize}
	\item Pre-procesado de los datos: Durante esta etapa se ejecutan varios flujos para limpieza de datos, en función de los requerimientos de cada algoritmo implementado. Entre estos flujos se encuentran el ajuste de tipo de columna, para llevar a cabo modificaciones que adecúen los datos al tipo de información que deberían representar; el manejo de columnas de tipo \textit{string}, para eliminar las inconsistencias que se pueden presentar, como caracteres extraños y el uso indistinto de mayúsculas y minúsculas en un mismo valor, así como la eliminación de valores únicos por columna; el manejo de valores faltantes, donde los atributos numéricos y categóricos se sustituyen por la media y moda respectivamente; el tratamiento de valores numéricos, para la conversión de valores nominales en numéricos, y proceder a su normalización, empleando el método Min-Max; y la discretización de variables numéricas, utilizando el método Decimal Scaling.
	\item Entrenamiento de modelos y evaluación: Implica la ejecución de los algoritmos previamente mencionados. En cada cual se ejecuta su evaluación para ser comparados posteriormente, en base a las métricas exactitud y AUC.
	\item Graficado de resultados y selección del modelo: Los resultados obtenidos en la fase anterior se grafican y  el usuario tiene la posibilidad de elegir el que satisfaga sus necesidades. En esta visualización se muestra un grafico de barras con las métricas mencionadas y una curva ROC , para evaluar el rendimiento de los modelos de clasificación. 
\end{itemize}

En su configuración se encuentran la selección de la columna objetivo, los caracteres especiales a eliminar durante el pre-procesado de \textit{string}, el umbral de valores únicos por columna y de valores perdidos permitidos, el porcentaje de partición mediante un muestreo estratificado y la selección de los algoritmos de clasificación que se emplearán. Cabe destacar que este componente permite la visualización y modificación de los flujos implementados. No obstante, a pesar de estar enfocado en el pre-procesado, presenta algunas deficiencias en esta tarea:
\begin{itemize}
	\item Discretización: Este proceso se encuentra estático, es decir, solamente se ejecuta un método de discretización con el nodo AutoBinner, el cual presenta otros métodos que pueden tener un mejor funcionamiento en función de los datos y modelos; incluso, existe otro nodo con el método CAIM para la discretización. Además, esta tarea solo se encuentra implementada en el modelo ID3, cuando C4.5, al ser un árbol de decisión, también trabaja y, a su vez, tolera mejor los datos discretizados.
	\item Normalización: Al igual que la discretización, esta se encuentra de forma estática, cuando en el nodo Normalizer empleado, presenta tres métodos para normalizar. Por otra parte, en este componente solamente está presente la normalización para el modelo Redes Neuronales por Retropropagación; sin embargo, los modelos PNN y SVM, aunque no lo tienen como requisito en la herramienta KNIME, tienen mejor desempeño con los datos normalizados.
	\item Imputación de valores faltantes: El método empleado es la sustitución por la media, en caso de los atributos numéricos, y la sustitución del valor más frecuente, en caso de los valores nominales. Este enfoque, sin estar erróneo, puede sustituirse por la aplicación de métodos de imputación más avanzados, como lo son los que emplean algoritmos de Aprendizaje Automático.
	\item Tratamiento de valores únicos por columna: La interpretación a esta tarea es que, dado un número de valores únicos que existan en una columna de datos nominales, si al contarlos superan este umbral, se eliminen estos valores. No obstante, en este componente se implementa de forma errónea esta tarea, dado que en realidad se cuenta la cantidad de valores distintos que puede tomar un atributo y, si esta cantidad supera a la indicada por el usuario, se elimina la columna. Tras el análisis efectuado en la sección \ref{alta-cardinalidad}, se puede concluir que ambos enfoques afectan el desempeño del modelo.
\end{itemize}
De igual forma, la optimización de hiperparámetros no fue implementada en esta solución, siendo una de las tareas más importantes en AutoML para mejorar el rendimiento de los algoritmos empleados.

\subsection{Análisis comparativo de implementaciones para AutoML en KNIME}
Tras analizar de forma independiente cada uno de estas implementaciones, en la tabla \ref{tab:comparacion-knime} se resumen las mismas, evaluando aspectos fundamentales para su modificación.  La disponibilidad de documentación desempeña un papel crucial, ya que la existencia de recursos claros y accesibles puede determinar la capacidad de los usuarios para comprender y aprovechar al máximo la herramienta. Además, la posibilidad de modificación y adaptación a necesidades específicas es un factor clave, permitiendo a los usuarios personalizar la herramienta según sus requerimientos. El nivel de automatización impacta directamente en la eficiencia del proceso de modelado, lo que es esencial para ahorrar tiempo y recursos. Sin embargo, es importante destacar que todas estas implementaciones presentan desventajas y limitaciones propias, que deben ser consideradas cuidadosamente al tomar una decisión informada sobre la elección de una solución.

% Please add the following required packages to your document preamble:
% \usepackage{booktabs}
% \usepackage{graphicx}
\begin{table}[H]
	\centering
	\caption{Comparativa de las implementaciones para AutoML en KNIME}
	\label{tab:comparacion-knime}
	\begin{spacing}{1.2}
	\resizebox{\textwidth}{!}{%
		\begin{tabular}{@{}lllll@{}}
			\toprule
			Herramienta                                                                                 & \begin{tabular}[c]{@{}l@{}}Disponibilidad de\\ documentación\end{tabular} & \begin{tabular}[c]{@{}l@{}}Nivel de \\ automatización\end{tabular} & \begin{tabular}[c]{@{}l@{}}Posibilidad de\\ modificación\end{tabular} & Desventajas                                                                                                                                                                                                                                 \\ \midrule      \addlinespace[10pt]
			Extensión H2O                                                                               & Limitada                                                                  & Baja                                                               & Media                                                                 & \begin{tabular}[c]{@{}l@{}}- Es necesaria la intervención humana\\ para la implementación de flujos.\\ - No contiene posibilidad de pre-procesado.\end{tabular}                                                                             \\ \addlinespace[10pt]
			Componente AutoML                                                                           & Limitada                                                                  & Alta                                                               & Baja                                                                  & \begin{tabular}[c]{@{}l@{}}- Las actividades de AutoML están \\ predeterminadas.\\ - No se toma en cuenta el algoritmo ni \\ las características de los datos para el \\ pre-procesado.\\ - Funciona en formato de caja negra.\end{tabular} \\ \addlinespace[10pt]
			\begin{tabular}[c]{@{}l@{}}Componente AutoML \\  Clasificación\\ (pre-procesado)\end{tabular} & Asequible                                                                 & Alta                                                               & Alta                                                                  & \begin{tabular}[c]{@{}l@{}}- No esta implementada la optimización \\ de hiperparámetros.\\ - Algunas actividades del pre-procesado\\ están estáticas.\end{tabular}                                                                          \\   \addlinespace[10pt]   \bottomrule
		\end{tabular}%
    	}
		\end{spacing}
\end{table}

Resulta evidente que lo mas factible es continuar el desarrollo del componente AutoML Clasificación (pre-procesado), para lo cual se propone la automatización total del pre-procesado de los datos y la implementación de la optimización de hiperparámetros como parte esencial para mejorar el rendimiento de los modelos de aprendizaje automático, ahorrar tiempo y recursos, aumentar la robustez, adaptarse a cambios en los datos y avanzar en la investigación y el desarrollo de algoritmos.\\
Durante el transcurso de esta investigación, el componente AutoML de KNIME fue actualizado, a la fecha del 20 de octubre de 2023 \citep{KNIME2023-11}. Estas actualizaciones incluyen algunas mejoras, donde la más relevante es el acceso a los flujos implementados y su modificación. No obstante, tras analizar a más detalle el funcionamiento del componente, se encuentra que se mantiene la predeterminación de las tareas de AutoML. Por ejemplo, la estrategia de optimización de hiperparámetros está establecida a Búsqueda Aleatoria, y para cambiar esta configuración, la única posibilidad es que el usuario se adentre en los flujos implementados; siendo el mismo problema en la etapa del pre-procesado con las actividades de discretización, normalización y manejo de valores faltantes. Por otra parte, el pre-procesado específico de cada algoritmo no se toma en cuenta, lo cual es fundamental en la etapa de preparación de los datos. Asimismo, no se incluyen hiperparámetros que si se toman en cuenta en el desarrollo de esta investigación.


\begin{comment}
	\section{Componente KNIME de AutoML para el pre-procesado en tareas de clasificación}\label{epig:componente-ernesto}
	Con tal objetivo, en \citep{Carrazana2022} se desarrolla un componente KNIME de AutoML para el pre-procesado en tareas de clasificación. Este, a partir de un conjunto de datos y una columna objetivo, ejecuta diferentes flujos de pre-procesado, en aras de cumplir con los requisitos de los diferentes algoritmos de clasificación, siendo capaz de entrenarlos y probarlos, para posteriormente puntuar y graficar los resultados \citep{Carrazana2022}. En el anexo \ref{anex:flujo-automl-componente} se muestra el flujo KNIME del Componente AutoML Clasificación (pre-procesado). \\
	Este componente está enfocado en el pre-procesado de datos, donde se desarrollaron subcomponentes enfocados en la realización de las tareas de pre-procesamiento de datos numéricos, \textit{string}, valores faltantes y el ajuste de tipos de columna. Como se observa en el anexo \ref{anex:flujo-automl-componente}, se aplican los algoritmos de clasificación ID3, C4.5, CART, Redes Neuronales por Retropropagación (RProp), Redes Neuronales Probabilísticas (PNN) y Máquina de Soporte Vectorial (SVM). Cada uno de estos requiere un tipo de pre-procesado diferente, acorde a los tipos de datos con los que trabaja. \\
	
	En los siguientes epígrafes se profundiza en estas dos tareas de vital importancia para el AutoML: el pre-procesado y la optimización de hiperparámetros. 
\end{comment}




\section{Métodos para la evaluación en clasificación}
En el campo de la clasificación, existen varios métodos de evaluación que se utilizan para comparar diferentes algoritmos o enfoques en esta tarea. Estos métodos ayudan a medir el rendimiento y la eficacia de los modelos de clasificación, así como a tomar decisiones informadas sobre cuál es el mejor enfoque para una tarea específica. Cuando se trabaja en problemas de clasificación con atributos numéricos, es fundamental contar con una base de datos de prueba adecuada. Una buena base de datos debe ser representativa de los datos del mundo real, y contener una variedad de instancias y etiquetas de clase para evaluar el rendimiento de los algoritmos de clasificación.

\subsection{Bases de datos de prueba}
Las bases de datos de prueba son conjuntos de datos creados para ayudar a los desarrolladores a probar y depurar aplicaciones de bases de datos, sin tener que utilizar datos reales y confidenciales. Estas bases de datos contienen datos ficticios, pero siguen la estructura de una base de datos real, lo que permite a los desarrolladores probar la funcionalidad de la aplicación sin preocuparse por dañar datos importantes o comprometer la privacidad de los usuarios. Kaggle Datasets\footnote{https://www.kaggle.com/datasets} y UCI Machine Learning Repository\footnote{https://archive.ics.uci.edu/datasets} son dos de los repositorios en línea más populares para conjuntos de datos de prueba y de aprendizaje automático. \\
Kaggle Datasets es un sitio web de aprendizaje automático que ofrece una amplia variedad de conjuntos de datos de muestra, desde datos meteorológicos hasta datos de redes sociales. Los usuarios pueden buscar entre miles de conjuntos de datos y también pueden contribuir con los propios. Kaggle también tiene una comunidad de científicos de datos y aprendizaje automático, que pueden proporcionar comentarios y ayudar a los usuarios a mejorar sus modelos de aprendizaje automático. \\
Por otro lado, el UCI Machine Learning Repository es un repositorio de conjuntos de datos de muestra para aprendizaje automático, minería de datos y otras aplicaciones de análisis de datos. El repositorio fue creado por la Universidad de California, Irvine, y contiene una amplia gama de conjuntos de datos, desde reconocimiento de voz hasta predicción de precios de viviendas. Los usuarios pueden descargar los conjuntos de datos de forma gratuita y utilizarlos para probar y desarrollar sus modelos de aprendizaje automático. \\
Ambos repositorios ofrecen una amplia variedad de conjuntos de datos, lo que los hace ideales para desarrolladores, estudiantes y profesionales de la ciencia de datos que buscan mejorar sus habilidades en el modelado de bases de datos y en la creación de modelos de aprendizaje automático precisos y efectivos. Por tal motivo, se emplearán ambos repositorios para la obtención de bases de datos para las pruebas que se realizarán en esta investigación.

% Please add the following required packages to your document preamble:
% \usepackage{booktabs}
% \usepackage{graphicx}
\begin{table}[H]
	\centering
	\caption{Bases de datos empleadas en la experimentación}
	\label{tab:bd-cap1}
	\begin{spacing}{1.2}
	\resizebox{\textwidth}{!}{%
		\begin{tabular}{@{}ccccccc@{}}
			\toprule
			Base de datos   & \#Instancias & \begin{tabular}[c]{@{}c@{}}\#Atributos\\ (cuantitativos, cualitativos)\end{tabular} & Clases                                                                      & \%Clases                                                  & Valores perdidos & Alta cardinalidad \\ \midrule
			\textsc{Human resources} & 19 158       & 14 (3, 11)                                                                          & \begin{tabular}[c]{@{}c@{}}yes \\  no\end{tabular}                          & \begin{tabular}[c]{@{}c@{}}24.93\%\\ 75.07\%\end{tabular} & si               & si                \\  \addlinespace[10pt]
			\textsc{Cancer data }    & 569          & 32 (31, 1)                                                                          & \begin{tabular}[c]{@{}c@{}}M \\ B\end{tabular}                              & \begin{tabular}[c]{@{}c@{}}37.26\%\\ 62.76\%\end{tabular} & no               & no                \\  \addlinespace[10pt]
			\textsc{Census income}   & 32 561       & 15 (6, 9)                                                                           & \begin{tabular}[c]{@{}c@{}}\textless{}=50k\\ \textgreater{}50k\end{tabular} & \begin{tabular}[c]{@{}c@{}}75.92\%\\ 24.08\%\end{tabular} & si               & no                \\ \bottomrule
		\end{tabular}%
	}
	\end{spacing}
\end{table}

La tabla \ref{tab:bd-cap1} resume las bases de datos empleados en este estudio. Muestra, para cada una, el número de instancias (\#Instancias), la cantidad de atributos (\#Atributos), número de atributos cuantitativos y cualitativos, distribución de atributos de clase y la ausencia o presencia de valores perdidos y atributos con alta cardinalidad.

\subsection{Métricas}
Al comparar algoritmos de clasificación, hay varias métricas que se utilizan para evaluar y comparar su rendimiento \citep{geron2022hands}, \citep{hastie2009elements}. A continuación, se presentan algunas de las métricas más comunes, empleadas en esta investigación:
\begin{itemize}
	\item Exactitud (Accuracy): Mide la proporción de todas las predicciones correctas realizadas por el modelo en relación con el tamaño total del conjunto de datos. En otras palabras, mide cuántas de todas las instancias (positivas y negativas) se han clasificado correctamente.. Sin embargo, la exactitud puede ser engañosa cuando hay desequilibrio de clases en el conjunto de datos.
	\item Precisión (Precision): La precisión es una métrica que evalúa la proporción de predicciones positivas hechas por un modelo que son verdaderamente positivas. Es decir, mide cuántas de las instancias clasificadas como positivas por el modelo realmente pertenecen a la clase positiva. 
	\item Exhaustividad (Recall): La exhaustividad, también conocida como sensibilidad o recall, es la proporción de ejemplos positivos que se clasifican correctamente en relación con el total de ejemplos positivos en el conjunto de datos. Mide la capacidad del modelo para identificar correctamente los ejemplos positivos.
	\item Puntuación F1 (F1 Score): La puntuación F1 es la media armónica de la precisión y la exhaustividad. Proporciona una medida equilibrada del rendimiento del modelo y es especialmente útil cuando hay un desequilibrio entre las clases.
	\item Matriz de confusión (Confusion Matrix): La matriz de confusión es una tabla que muestra el número de predicciones correctas e incorrectas realizadas por un modelo de clasificación. A partir de la matriz de confusión, se pueden calcular métricas como la precisión, la exhaustividad y la puntuación F1.
	\item Curva ROC y área bajo la curva (ROC Curve y AUC): La curva ROC es una representación gráfica del rendimiento del modelo a diferentes niveles de umbral. Muestra la tasa de verdaderos positivos (sensibilidad) en función de la tasa de falsos positivos (1 - especificidad). El área bajo la curva (AUC) es una métrica que resume la curva ROC y proporciona una medida del rendimiento general del modelo.
\end{itemize}

\section{Conclusiones parciales}

% Cada conclusión tiene que estar sustentada en el cuerpo del capítulo.

A partir de lo estudiado en este capítulo, se llega a las siguientes conclusiones:

\begin{itemize}
	\item El Aprendizaje Automático es una técnica que permite a las computadoras aprender a partir de datos, sin necesidad de ser programadas explícitamente.
	\item El proceso de descubrimiento de conocimiento en bases de datos (KDD) es un proceso iterativo que consiste en varias etapas, incluyendo la selección de datos, la limpieza de datos, la transformación de datos y la minería de datos.
	\item La Minería de Datos es el proceso de descubrir patrones y relaciones interesantes en grandes conjuntos de datos, utilizando técnicas de aprendizaje automático, estadísticas y visualización de datos. Algunas de las técnicas utilizadas en la Minería de Datos incluyen la clasificación, la agrupación, la regresión y la asociación.
	\item La Clasificación es una técnica de aprendizaje automático que se utiliza para predecir la etiqueta o clase de un objeto a partir de un conjunto de características.
	\item El AutoML se puede utilizar para mejorar la eficiencia y la precisión del proceso de modelado, reducir la necesidad de conocimientos especializados y permitir a los usuarios enfocarse en la interpretación de los resultados. Las etapas del AutoML incluyen la selección automática de algoritmos, el preprocesamiento de datos, la optimización de hiperparámetros y la evaluación automática del modelo.
	\item El pre-procesamiento de datos es una etapa crítica en el proceso de modelado, ya que los datos deben limpiarse, integrarse y transformarse antes de ser utilizados por los algoritmos de aprendizaje automático.
	\item Algunas de las tareas comunes del pre-procesado de datos incluyen la eliminación de valores atípicos, el manejo de datos faltantes, la discretización y la normalización de datos numéricos.
	\item La discretización es una técnica utilizada para transformar datos numéricos en datos categóricos.
	\item Los hiperparámetros son ajustes que se realizan en los algoritmos de aprendizaje automático para mejorar su rendimiento. La optimización de hiperparámetros implica encontrar la combinación óptima de valores para los hiperparámetros.
	\item KNIME es una herramienta de minería de datos de código abierto que permite a los usuarios crear y ejecutar flujos de trabajo de análisis de datos.
	\item Se implementó un componente KNIME en \citep{Carrazana2022} que brinda soporte para tareas de AutoML, enfocándose en la etapa de pre-procesado, en el cual se basa el desarrollo de los componentes en este proyecto.
\end{itemize}


%Una vez terminado el capítulo se arriban a las siguientes conclusiones:

%\begin{enumerate}
%	\setlength\itemsep{0em}
%	\item Una conclusión necesaria aquí son los requisitos principales que debe cumplir la solución propuesta.
%	\item Otra conclusión es la inexistencia de una solución que brinde cumplimiento a los requisitos planteados.
%	\item Finalmente, cuáles son las tecnologías seleccionadas y su justificación
%\end{enumerate}

\pagebreak

	\chapter{Propuesta de modificaciones al componente de AutoML para pre-procesado}\label{chap:2}
En el presente capítulo, se propone una serie de modificaciones destinadas a mejorar el componente AutoML Clasificación (pre-procesado). Estas modificaciones se centran en el preprocesamiento de datos, incluyendo la automatización de tareas como la discretización, normalización y el tratamiento de valores únicos, valores de alta cardinalidad y valores faltantes. Además, se explorará la inclusión de técnicas de Optimización de Hiperparámetros (HPO) para encontrar configuraciones óptimas para los modelos. El enfoque principal de esta propuesta es la integración de estas dos facetas con el componente AutoML existente para clasificación. A lo largo de este capítulo, se describirá cómo estas modificaciones buscan mejorar la eficiencia y precisión del proceso de AutoML, creando modelos de clasificación más sólidos y adaptados a las necesidades específicas de los datos.

\section{Modificaciones en el pre-procesado}
Con el fin de abordar las problemáticas presentadas en el epígrafe \ref{epig:componente-ernesto}, se llevaron a cabo una serie de modificaciones sustanciales en el proceso de pre-procesamiento de datos. Estas adaptaciones se centraron en la automatización de la discretización y normalización, así como en el tratamiento de valores únicos, valores de alta cardinalidad y valores faltantes. En lo que respecta a la discretización y normalización, se implementaron técnicas automatizadas para garantizar una uniformidad en la escala y distribución de los datos, lo que contribuye a mejorar la precisión del modelado. Para abordar los valores únicos y de alta cardinalidad, se desarrollaron estrategias específicas que permitieron una gestión más efectiva de estas categorías, evitando la pérdida de información esencial y reduciendo el riesgo de sobreajuste. Por último, se implementaron métodos especializados para tratar los valores faltantes, asegurando que los datos incompletos no comprometieran la calidad del análisis. En las secciones siguientes, se detalla el funcionamiento y modelado de estas modificaciones con mayor profundidad.


\subsection{Discretización de variables numéricas} 
La discretización de variables numéricas se utiliza para convertir datos continuos en datos discretos, lo que permite que los algoritmos de aprendizaje automático puedan procesarlos y analizarlos adecuadamente. La discretización también puede mejorar la precisión de los modelos de aprendizaje automático al reducir el ruido en los datos y hacer que los patrones sean más fáciles de detectar. En aras de automatizar este proceso, se propone el subcomponente \textit{Discretizer} (Figura \ref{fig:subcomp-disc}).

\begin{figure}[H]
	\centering
	\includegraphics[width=0.15\linewidth]{"figuras/capi 2/subcomp-disc"}
	\caption[Subcomponente Discretizer]{Subomponente \textit{Discretizer}}
	\label{fig:subcomp-disc}
\end{figure}

El subcomponente \textit{Discretizer} toma en su puerto de entrada los datos en formato tabular. Dichos datos, de tipo numéricos, son procesados por varios algoritmos de discretización y, como puerto de salida, se obtienen discretizados acorde al método de discretización con mejor precisión. Tiene una única restricción: no pueden tener valores perdidos como entrada. El diagrama de flujo de la figura \ref{fig:discretizacion} expone el flujo general del componente \textit{Discretizer}. El flujo KNIME correspondiente está presente en el anexo \ref{anex:flujo-disc-comp-id3}.

\begin{figure}[H]
	\centering
	\includegraphics[width=1\linewidth]{"figuras/capi 2/preprocesado/discretizacion.drawio"}
	\caption{Diagrama de flujo general de Discretizer}
	\label{fig:discretizacion}
\end{figure}

Para la discretización se emplean los nodos presentes en la figura \ref{fig:discretizacion-nodos}. El nodo \textit{Auto-Binner} contiene tres métodos de discretización: \textit{Equal-Width, Equal-Frequency} y \textit{Quantile-Based}; mientras el nodo \textit{CAIM Binner} contiene el método \textit{CAIM}. Para la elección de los \textit{k} intervalos se emplea la Regla de Sturges, descrita en la sección \ref{sub-epigrafe-disc}. El nodo \textit{CAIM Binner} no requiere una configuración en específico, y para el método Quantile-Based se escogieron los cuantiles por defecto, es decir 0.0, 0.25, 0.5, 0.75 y 1.0.

\begin{figure}[H]
	\centering
	\begin{subfigure}[b]{0.25\linewidth}
		\centering
		\includegraphics[width=0.5\linewidth]{"figuras/capi 2/auto-binner-nodo"}
		\caption{Nodo \textit{Auto-Binner}}
		\label{fig:auto-binner-nodo}
	\end{subfigure}
	\hspace{1.5cm}
	\begin{subfigure}[b]{0.25\linewidth}
		\centering
		\includegraphics[width=0.5\linewidth]{"figuras/capi 2/caim-binner-nodo"}
		\caption{Nodo \textit{CAIM Binner}}
		\label{fig:caim-binner-nodo}
	\end{subfigure}
	\caption{Nodos empleados para la discretización}
	\label{fig:discretizacion-nodos}
\end{figure}

Tras discretizar el conjunto de datos con cada método correspondiente, se aplica el modelo en cuestión a cada uno de los datos resultantes, con el objetivo de compararlos en cuanto a precisión y Cohen's Kappa, y escoger el algoritmo con mejor desempeño. 

\subsection{Normalización de variables numéricas}
La normalización son proporciones sin unidades de medida (adimensionales o invariantes de escala) que nos permiten poder comparar elementos de distintas variables y distintas unidades de medida. Esta es necesaria para cambiar los valores de las columnas numéricas del conjunto de datos para usar una escala común, sin distorsionar las diferencias en los intervalos de valores ni perder información. La normalización es fundamental para que algunos algoritmos modelen los datos correctamente. \\
En KNIME es posible implementar la normalización a partir del nodo "Normalizer", presente en la FIGURA X. En este nodo se encuentran tres métodos para normalizar, a elección del usuario: Decimal Scaling, Z-Score y Min-Max. En aras de automatizar este proceso, se propone el subcomponente "Normalizer", de igual nombre al nodo nativo de KNIME, en donde se encuentra la comparación de los métodos anteriormente mencionados en función de un modelo predeterminado. En la figura \ref{fig:normalizacion} se presenta el diagrama de flujo de este subcomponente. 

\begin{figure}[H]
	\centering
	\includegraphics[width=1\linewidth]{"figuras/capi 2/preprocesado/normalizacion.drawio"}
	\caption{Diagrama de flujo para la normalización}
	\label{fig:normalizacion}
\end{figure}

Siguiendo el mismo esquema del subcomponente "Discretizer", tras aplicar la normalización con los distintos métodos ofrecidos por la herramienta KNIME, se aplica el algoritmo de aprendizaje automatico que requiere los datos normalizados y posteriormente, se realiza la evaluación del desempeño de cada uno, acorde a la precisión y el Cohen's Kappa. Luego de escoger el mejor, se devuelve la tabla con los datos normalizados.


\subsection{Pre-procesado de variables de tipo \textit{string}}
Los valores nominales únicos, o categorías con un solo ejemplo en una columna, pueden parecer insignificantes a primera vista, pero su correcto manejo es esencial para evitar posibles problemas de calidad de datos y garantizar la integridad de los análisis. Por otro lado, los valores de alta cardinalidad, aquellos que se repiten con frecuencia y pueden ser numerosos, son cruciales para comprender tendencias y patrones en los datos. El Componente AutoML  Clasificación (pre-procesado) poseía un tratamiento erróneo de estos valores, al eliminar las columnas nominales que superaban un umbral determinado de categorías distintas. 
En aras de depurar estas inconsistencias, el diagrama de la figura \ref{fig:string-preprocs} muestra un nuevo flujo para el pre-procesado de String, donde la actividad de color amarillo expresa que se modifico la que anteriormente se encontraba en el componente; mientras que las actividades en verde reflejan las nuevas implementaciones.

\begin{figure}[H]
	\centering
	\includegraphics[width=0.95\linewidth]{"figuras/capi 2/preprocesado/string preprocs.drawio"}
	\caption{Diagrama de flujo para el pre-procesado de string}
	\label{fig:string-preprocs}
\end{figure}

A continuación se exponen los cambios realizados al componente String preprocs:
\begin{itemize}
	\item Eliminar valores únicos por columna:  Se enmienda el error antes expuesto, al modificar el subcomponente "filtrar valores únicos", implementando la eliminación de las columnas donde mas del 80\% de los valores son únicos.
	\item Reemplazar con otros: En este nuevo componente, los valores únicos en una columna que representan una minoría, son reemplazados por la categoría 'otros'.  Para ello, iterando por cada columna dentro de un ciclo, se calculan la frecuencia absoluta y relativa de cada atributo, y aquellos que representen menos del 1\% son los elegidos para la sustitución por la nueva categoría. 
	\item Codificar con One-Hot Encoding: Para el tratamiento de valores con alta cardinalidad, se emplea la codificación One-Hot, utilizando el nodo One To Many. Para esto se escogen los atributos que por defecto tienen, como mínimo, 15 categorías distintas. Dado que esta técnica produce gran dimensionalidad, se utilizan para su reducción los métodos
	\begin{enumerate}
		\item Filtrar por varianza, utilizando el nodo Low Variance Filter, donde las columnas creadas con una varianza menor a 0.01 son eliminadas; y 
		\item Filtrar por correlación,  utilizando el nodo Correlation Filter, donde las columnas con una correlación mayor a 0.9 son eliminadas.
	\end{enumerate}
\end{itemize}

El flujo KNIME de los nuevos subcomponentes se encuentran en los ANEXOS X y Y.


\subsection{Manejo de valores faltantes}
El tratamiento de valores faltantes en conjuntos de datos es un paso crítico en la preparación y análisis de datos. La presencia de datos faltantes puede afectar significativamente la calidad y la fiabilidad de cualquier análisis o modelo que se derive de ellos. Adicionalmente, algunos algoritmos requieren que no existan valores faltantes para su funcionamiento. En aras de mejorar la imputacion de estos valores, se propone modificar el subcomponente 'Valores faltantes', cuyo diagrama de flujo se presenta en la figura \ref{fig:valores-faltantes}.

\begin{figure}[H]
	\centering
	\includegraphics[width=1\linewidth]{"figuras/capi 2/preprocesado/valores faltantes.drawio"}
	\caption{Diagrama de flujo para el manejo de valores faltantes}
	\label{fig:valores-faltantes}
\end{figure}

La imputación de valores faltantes ha sido cambiada, por los motivos expuestos en la sección \ref{secc:mv}. En la figura \ref{fig:mv-imputation} se presenta el diagrama de flujo de la nueva implementación para la sustitución de estos valores. 

\begin{figure}[H]
	\centering
	\includegraphics[width=1\linewidth]{"figuras/capi 2/preprocesado/mv imputation"}
	\caption{Diagrama de flujo para la imputación de valores faltantes}
	\label{fig:mv-imputation}
\end{figure}

Primeramente se aplican en paralelo las técnicas de imputación KNNI y KMI, al terminar se aplica el algoritmo de aprendizaje automático en cuestión con los datos imputados por cada método y, finalmente, se comparan los resultados acorde a las métricas Precisión y Cohen's Kappa para devolver los datos con la mejor sustitución. En el ANEXO X se muestra el flujo KNIME de estas estrategias. 



\section{Componente AutoML Clasificación (Optimización de hiperparámetros)}
Antes de explorar en detalle las adaptaciones realizadas en cada modelo, es fundamental presentar el Componente AutoML Clasificación (Optimización de hiperparámetros), con el objetivo de facilitar el entendimiento en los epígrafes posteriores. 

\textbf{tooooodo lo que lleva con sus epígrafes y esa talla, esto es lo que estaba en las practicas}

El componente propuesto, presente en la figura \ref{fig:automl-componente-hpo}, para la optimización de hiperparámetros, contiene la siguiente configuración:
\begin{figure}[H]
	\centering
	\includegraphics[width=0.35\linewidth]{"figuras/capi 2/automl-componente-hpo"}
	\caption[Componente AutoML Clasificación (Optimización de Hiperparámetros)]{Componente \textit{AutoML Clasificación (Optimización de Hiperparámetros)}}
	\label{fig:automl-componente-hpo}
\end{figure}
\begin{enumerate}
	\item Puerto de entrada: recibe los datos de entrada en formato tabular.
	\item Elementos de la configuración:
	\begin{itemize}
		\item Columna objetivo: presenta las columnas de tipo \textit{string} que pueden fungir como columna objetivo.
		\item Estrategia de optimización de hiperparámetros: se selecciona la estrategia de optimización de hiperparámetros entre las disponibles (Random Search, Bayesian Optimization (TPE), Brute Force y Hillclimbing).
		\item Selección del número de subconjuntos de la validación cruzada: el valor introducido determina la cantidad de veces que se divide el conjunto de datos en subconjuntos de entrenamiento y prueba, durante el proceso de validación.
	\end{itemize}
	\item Puerto de salida: tabla de hiperparámetros optimizados, siguiendo la estrategia de optimización escogida.
\end{enumerate}

\subsection{Uso y configuración del componente AutoML Clasificación (Optimización de Hiperparámetros)}
A continuación, se define la secuencia de pasos para el correcto funcionamiento del componente \textit{AutoML Clasificación (Optimización de Hiperparámetros)}:

\begin{enumerate}
	\item Proporcionar el conjunto de datos al puerto de entrada (el componente marcará error en este punto, pues la configuración es obligatoria).
	\item Dar click derecho sobre el componente, seleccionar “Configure…”. 
	\item Seleccionar la configuración deseada (como se observa en el ejemplo de la figura \ref{fig:config-automl-hpo}). 
	\item Ejecutar el componente (inicialmente el puerto de salida se encuentra vacío).
	\item Dar click derecho sobre el componente y seleccionar “Interactive View: AutoML” (Fig3). 
\end{enumerate}

\begin{figure}[H]
	\centering
	\includegraphics[width=0.5\linewidth]{"figuras/capi 2/config-automl-hpo"}
	\caption[Ejemplo de configuración del componente AutoML (Optimización de Hiperparámetros)]{Ejemplo de configuración del componente \textit{AutoML (Optimización de Hiperparámetros)}}
	\label{fig:config-automl-hpo}
\end{figure}

\begin{figure}[H]
	\centering
	\includegraphics[width=0.5\linewidth]{"figuras/capi 2/automl-hpo-vista-salida"}
	\caption[Vista de la salida.]{Vista de la salida.}
	\label{fig:automl-hpo-vista-salida}
\end{figure}

\subsection{Requisitos y restricciones del componente AutoML Clasificación (Optimización de Hiperparámetros)}
El componente propuesto debe cumplir los siguientes requisitos funcionales:

\begin{itemize}
	\item RF1: El componente debe permitir seleccionar columna objetivo
	\item RF2: El componente debe permitir seleccionar estrategia de optimización de hiperparámetros.
	\item RF3: El componente debe permitir seleccionar cantidad de subconjuntos en la validación cruzada.
	\item RF4: El componente debe optimizar hiperparámetros y entrenar datos para Redes Neuronales de Retro propagación.
	\item RF5: El componente debe retornar tabla de hiperparámetros optimizados.
\end{itemize}

El componente propuesto presenta las siguientes restricciones para su funcionamiento:

\begin{itemize}
	\item Los datos de entrada deben encontrarse en formato tabular.
	\item La columna objetivo debe ser de tipo \textit{string}.
	\item Los datos de entrada deben ser de tipo numérico, a excepción de la columna objetivo.
	\item Los datos numéricos deben estar previamente normalizados.
\end{itemize}

\subsection{Modelación del componente AutoML Clasificación (Optimización de Hiperparámetros)}
El diagrama de flujo de la figura \ref{fig:diagrama-flujo-gral-comp-hpo} expone el flujo general del componente \textit{AutoML Clasificación (Optimización de Hiperparámetros)}.

\begin{figure}[H]
	\centering
	\includegraphics[width=0.7\linewidth]{"figuras/capi 2/diagrama-flujo-gral-comp-hpo"}
	\caption[Diagrama de flujo general del componente AutoML Clasificación (Optimización de Hiperparámetros)]{Diagrama de flujo general del componente \textit{AutoML Clasificación (Optimización de Hiperparámetros)}}
	\label{fig:diagrama-flujo-gral-comp-hpo}
\end{figure}

La configuración de la selección de parámetros es el primer paso, en el cual se crearán las variables que dictarán el comportamiento del flujo. Posteriormente, se ejecuta la optimización de hiperparámetros para RProp, donde recoge los resultados en una tabla y luego los grafica. El flujo KNIME correspondiente se evidencia en el Anexo \ref{aped.flujo-hpo-rprop}.

\subsubsection{Selección de parámetros}
Los parámetros que rigen el funcionamiento del componente propuesto, brindan al usuario una mayor personalización de la optimización de hiperparámetros, pues le ofrece la libertad de configurar múltiples factores claves de esta etapa. El diagrama de actividades de la figura \ref{fig:diagrama-act-selecc-param-hpo} expone el flujo para la selección de parámetros.

\begin{figure}[H]
	\centering
	\includegraphics[width=0.7\linewidth]{"figuras/capi 2/diagrama-act-selecc-param-hpo"}
	\caption[Diagrama de actividades de selección de parámetros]{Diagrama de actividades de selección de parámetros}
	\label{fig:diagrama-act-selecc-param-hpo}
\end{figure}

La selección de parámetros se realiza con los siguientes nodos de configuración, presentes en el repositorio base:

\begin{itemize}
	\item Seleccionar columna objetivo: se emplea el nodo \textit{Column Selection Configuration} (Figura \ref{fig:nodo-column-select-conf}), el cual recibe una tabla y devuelve el nombre de la columna seleccionada como variable de flujo. En este caso, presenta la configuración adicional para solo mostrar las columnas de tipo \textit{string}.
	\begin{figure}[H]
		\centering
		\includegraphics[width=0.15\linewidth]{"figuras/capi 2/nodo-column-select-conf"}
		\caption[Nodo Column Selection Configuration]{Nodo \textit{Column Selection Configuration}}
		\label{fig:nodo-column-select-conf}
	\end{figure}
	
	\item Seleccionar estrategia de optimización de hiperparámetros: la selección de la estrategia se lleva a cabo empleando el nodo \textit{Single Selection Configuration} (Figura \ref{fig:nodo-single-select-conf}). Devuelve la variable \texttt{strategy} con el valor seleccionado previamente.
	\begin{figure}[H]
		\centering
		\includegraphics[width=0.15\linewidth]{"figuras/capi 2/nodo-single-select-conf"}
		\caption[Nodo Single Selection Configuration]{Nodo \textit{Single Selection Configuration}}
		\label{fig:nodo-single-select-conf}
	\end{figure}
	
	\item Seleccionar número de subconjuntos en la validación cruzada: la selección de la cantidad de subconjuntos de partición de entrenamiento, se lleva a cabo con el nodo \textit{Integer Configuration} (Figura \ref{fig:nodo-int-conf}). Este devuelve la variable de flujo resultante de la selección, en este caso presenta la configuración para limitar el rango entre 5 y 10.
\end{itemize}
\begin{figure}[H]
	\centering
	\includegraphics[width=0.15\linewidth]{"figuras/capi 2/nodo-int-conf"}
	\caption[Nodo Integer Configuration]{Nodo \textit{Integer Configuration}}
	\label{fig:nodo-int-conf}
\end{figure}

\subsection{Optimización de hiperparámetros para RProp}
Las Redes Neuronales por Retro-propagación necesitan que todos los valores sean de tipo numéricos y estos se encuentren normalizados. Para el entrenamiento y prueba de las Redes Neuronales por Retro-propagación, se emplean los nodos \textit{RProp MLP Learner} y \textit{MultiLayerPerceptron Predictor} (Figura \ref{fig:nodos-rprop}) respectivamente.

\begin{figure}[H]
	\centering
	\includegraphics[width=0.4\linewidth]{"figuras/capi 2/nodos-rprop"}
	\caption[Nodos para entrenar y probar Redes Neuronales por retro propagación]{Nodos para entrenar y probar Redes Neuronales por retro-propagación}
	\label{fig:nodos-rprop}
\end{figure}

El diagrama de actividades de la figura \ref{fig:diagrama-act-proc-rprop-hpo}, expone el flujo para el procesamiento necesario para la ejecución del algoritmo Redes Neuronales por Retro-ropagación, con optimización de hiperparámetros.
\begin{figure}[H]
	\centering
	\includegraphics[width=0.7\linewidth]{"figuras/capi 2/diagrama-act-proc-rprop-hpo"}
	\caption[Diagrama de actividades para el procesamiento de RProp con HPO]{Diagrama de actividades para el procesamiento de RProp con HPO}
	\label{fig:diagrama-act-proc-rprop-hpo}
\end{figure}

Para llevar a cabo el procesamiento RProp se emplean los siguientes nodos: 
\begin{itemize}
	\item Ciclo para recorrer el rango de hiperparámetros: se emplean los nodos \textit{Parameter Optimization Loop Start} y \textit{Parameter Optimization Loop End} (Figura \ref{fig:nodos-param-opt-loop}). Ambos permiten guardar todas las iteraciones realizadas por el algoritmo con las diferentes combinaciones de hiperparámetros. Para su configuración, se eligieron como hiperparámetros el número de capas, la cantidad de neuronas y el número máximo de iteraciones (Figura \ref{fig:conf-nodo-param-loop}).
	\begin{figure}[H]
		\centering
		\includegraphics[width=0.5\linewidth]{"figuras/capi 2/nodos-param-opt-loop"}
		\caption[Nodos Parameter Optimization Loop Start y Parameter Optimization Loop End]{Nodos \textit{Parameter Optimization Loop Start} y \textit{Parameter Optimization Loop End}}
		\label{fig:nodos-param-opt-loop}
	\end{figure}
	
	\begin{figure}[H]
		\centering
		\includegraphics[width=0.6\linewidth]{"figuras/capi 2/conf-nodo-param-loop"}
		\caption[Configuración del nodo Parameter Optimization Loop Start]{Configuración del nodo \textit{Parameter Optimization Loop Start}}
		\label{fig:conf-nodo-param-loop}
	\end{figure}
	
	\item Ciclo para dividir el conjunto de datos: se emplean los nodos \textit{X-Partitioner} y \textit{X-Aggregator} (Figura \ref{fig:nodos-cross-val}), para dividir el conjunto de datos en \textit{k} particiones y realizar una validación cruzada, donde cada partición se utiliza como conjunto de prueba una vez y las otras \textit{k-1} particiones se utilizan como conjunto de entrenamiento.
	\begin{figure}[H]
		\centering
		\includegraphics[width=0.4\linewidth]{"figuras/capi 2/nodos-cross-val"}
		\caption[Nodos X-Partitioner y X-Aggregator]{Nodos \textit{X-Partitioner} y \textit{X-Aggregator}}
		\label{fig:nodos-cross-val}
	\end{figure}
	
	\item Calcular la exactitud: se emplea el nodo \textit{Scorer} (Figura \ref{fig:nodo-scorer}), el cual recibe la predicción y la columna objetivo en una tabla para la evaluación. 
	\begin{figure}[H]
		\centering
		\includegraphics[width=0.15\linewidth]{"figuras/capi 2/nodo-scorer"}
		\caption[Nodo Scorer]{Nodo \textit{Scorer}}
		\label{fig:nodo-scorer}
	\end{figure}
	
	\item Graficar: se emplea el nodo \textit{Table View} (Fig13), capaz de visualizar la tabla de los hiperparámetros con mejor resultado.
	\begin{figure}[H]
		\centering
		\includegraphics[width=0.15\linewidth]{"figuras/capi 2/nodo-table-view"}
		\caption[Nodo Table View]{Nodo \textit{Table View}}
		\label{fig:nodo-table-view}
	\end{figure}
	
\end{itemize}


\subsection{Optimización de hiperparámetros para PNN}
blablabla
blablabla

\subsection{Optimización de hiperparámetros para SVM}

blablabla

\subsection{Optimización de hiperparámetros para Random Forest}
blablabla






\section{Modelación de nueva versión del componente AutoML Clasificación}

\begin{itemize}
	\item componente de ernesto
	\item diagrama nuevo
	\item cambios hechos en la personalización: random forest, quitado lo de valores únicos por columna, agregar la estrategia de optimización
\end{itemize}

\subsection{Procesado de ID3}
blablabla


\begin{figure}[H]
	\centering
	\includegraphics[width=1\linewidth]{"figuras/capi 2/modelos/procesado id3.drawio"}
	\caption{Diagrama de flujo del procesamiento del modelo ID3}
	\label{fig:procesado-id3}
\end{figure}

blablabla 



\subsection{Procesado para C4.5 y CART}
blablabla

\begin{figure}[H]
	\centering
	\includegraphics[width=1\linewidth]{"figuras/capi 2/modelos/procesado c4pt5.drawio"}
	\caption{Diagrama de flujo del procesamiento del modelo C4.5}
	\label{fig:procesado-c4pt5}
\end{figure}

blablabla

\begin{figure}[H]
	\centering
	\includegraphics[width=1\linewidth]{"figuras/capi 2/modelos/procesado cart.drawio"}
	\caption{Diagrama de flujo del procesado de CART}
	\label{fig:procesado-cart}
\end{figure}

blablabla

\subsection{Procesamiento para Redes Neuronales por Retropropagación}
blablabla

 \begin{figure}[H]
	\centering
	\includegraphics[width=1\linewidth]{"figuras/capi 2/modelos/procesado rprop.drawio"}
	\caption{Diagrama de flujo del procesamiento de RProp}
	\label{fig:procesado-rprop}
\end{figure}

blablabla


\subsection{Procesamiento para Redes Neuronales Probabilísticas}
blablabla
\begin{figure}[H]
	\centering
	\includegraphics[width=1\linewidth]{"figuras/capi 2/modelos/procesado pnn.drawio"}
	\caption{Diagrama de flujo para el procesado de PNN}
	\label{fig:procesado-pnn}
\end{figure}

blablabla


\subsection{Procesamiento para SVM}
blablabla
\begin{figure}[H]
	\centering
	\includegraphics[width=1\linewidth]{"figuras/capi 2/modelos/procesado svm.drawio"}
	\caption{Diagrama de flujo del procesamiento de SVM}
	\label{fig:procesado-svm}
\end{figure}

blablabla



\subsection{Procesamiento para Random Forest}

blablabla

\begin{figure}[H]
	\centering
	\includegraphics[width=1\linewidth]{"figuras/capi 2/modelos/procesado rf.drawio"}
	\caption{Diagrama de flujo del procesamiento de Random Forest}
	\label{fig:procesado-rf}
\end{figure}

blablabla


\section{Conclusiones parciales}
A partir de lo analizado en este capítulo, se arriba a las siguientes conclusiones:
\begin{itemize}
	\item Se obtiene el diseño de un subcomponente para la discretización de variables numéricas.
	\item Para la discretización, se implementan los métodos Equal-Width, Equal-Frequency, Quantile-Based y CAIM.
	\item La implementación de los métodos de discretización se realiza con los nodos nativos de KNIME \textit{Auto-Binner} y \textit{CAIM Binner}.
	\item Se obtiene el diseño del componente AutoML Clasificación (Optimización de hiperparámetros).
	\item Se describe el uso y  los diferentes requisitos y restricciones del componente AutoML Clasificación (Optimización de hiperparámetros).
	\item Se define el rango de hiperparámetros (Iteraciones, capas y número de neuronas) del algoritmo Rprop.
\end{itemize}

\pagebreak


	\chapter{Validación de soluciones propuestas al componente de AutoML}\label{chap:3}
En este capítulo .....


\section{Pruebas de caja negra a subcomponentes para el pre-procesado}

... muelita linda

\subsection{Caso de prueba al subcomponente \textit{Discretizer}}

% Please add the following required packages to your document preamble:
% \usepackage{graphicx}
\begin{table}[H]
	\centering
	\begin{spacing}{1.3}
	\resizebox{\columnwidth}{!}{%
		\begin{tabular}{|llll|}
			\hline
			\multicolumn{4}{|l|}{Caso de prueba}                                                                                                                                                                                                                                                                                                                                                                                                                                         \\ \hline
			\multicolumn{1}{|l|}{\begin{tabular}[c]{@{}l@{}}Objetivo de la \\ prueba\end{tabular}}    & \multicolumn{3}{l|}{Comprobar la efectividad de las transformaciones al discretizar las variables numéricas}                                                                                                                                                                                                                                                                     \\ \hline
			\multicolumn{1}{|l|}{\begin{tabular}[c]{@{}l@{}}Descripción de \\ la prueba\end{tabular}} & \multicolumn{3}{l|}{\begin{tabular}[c]{@{}l@{}}Se debe proporcionar una tabla con variables numéricas al componente \textit{Discretizer} y \\ evaluar la tabla resultante\end{tabular}}                                                                                                                                                                                                   \\ \hline
			\multicolumn{1}{|l|}{Condiciones}                                                         & \multicolumn{3}{l|}{\begin{tabular}[c]{@{}l@{}}1. Debe estar presente la columna objetivo para la clasificación.\\ 2. La columna objetivo debe ser de tipo nominal.\end{tabular}}                                                                                                                                                                                                \\ \hline
			\multicolumn{4}{|l|}{Combinaciones de valores de entrada}                                                                                                                                                                                                                                                                                                                                                                                                                    \\ \hline
			\multicolumn{1}{|l|}{CP}                                                                  & \multicolumn{1}{l|}{Escenario}                                                                                     & \multicolumn{1}{l|}{Resultado esperado}                                                                                                 & Resultado real                                                                                                    \\ \hline
			\multicolumn{1}{|l|}{CP1}                                                                 & \multicolumn{1}{l|}{\begin{tabular}[c]{@{}l@{}}Se proporciona una \\ tabla con atributos\\ numéricos\end{tabular}} & \multicolumn{1}{l|}{\begin{tabular}[c]{@{}l@{}}Se discretizan los atributos \\ numéricos con cada método\end{tabular}}                  & \begin{tabular}[c]{@{}l@{}}Se discretizan los atributos \\ numéricos con cada método\end{tabular}                 \\ \hline
			\multicolumn{1}{|l|}{CP2}                                                                 & \multicolumn{1}{l|}{\begin{tabular}[c]{@{}l@{}}Se proporciona una\\ tabla con atributos\\ numéricos\end{tabular}}  & \multicolumn{1}{l|}{\begin{tabular}[c]{@{}l@{}}Se comparan los discretizadores \\ acorde al algoritmo de ML\end{tabular}}               & \begin{tabular}[c]{@{}l@{}}Se comparan los discretizadores \\ acorde al algoritmo de ML\end{tabular}              \\ \hline
			\multicolumn{1}{|l|}{CP3}                                                                 & \multicolumn{1}{l|}{\begin{tabular}[c]{@{}l@{}}Se proporciona una\\ tabla con atributos \\ numéricos\end{tabular}} & \multicolumn{1}{l|}{\begin{tabular}[c]{@{}l@{}}Se devuelven los datos \\ discretizados acorde a los \\ resultados del CP2\end{tabular}} & \begin{tabular}[c]{@{}l@{}}Se devuelven los datos\\ discretizados acorde a los \\ resultados del CP2\end{tabular} \\ \hline
		\end{tabular}%
	}
	\end{spacing}
	\caption{Caso de prueba al componente \textit{Discretizer}}
	\label{tab:cp-disc}
\end{table}



\subsection{Caso de prueba al subcomponente \textit{String preprocs}}

% Please add the following required packages to your document preamble:
% \usepackage{graphicx}
\begin{table}[H]
	\centering
	\begin{spacing}{1.3}
	\resizebox{\columnwidth}{!}{%
		\begin{tabular}{|llll|}
			\hline
			\multicolumn{4}{|l|}{Caso de prueba}                                                                                                                                                                                                                                                                                                                                                                                                                                                                     \\ \hline
			\multicolumn{1}{|l|}{\begin{tabular}[c]{@{}l@{}}Objetivo de la \\ prueba\end{tabular}}    & \multicolumn{3}{l|}{\begin{tabular}[c]{@{}l@{}}Comprobar la efectividad de las transformaciones a las columnas de tipo \textit{string}\\ y evaluar la tabla resultante\end{tabular}}                                                                                                                                                                                                                                  \\ \hline
			\multicolumn{1}{|l|}{\begin{tabular}[c]{@{}l@{}}Descripción de \\ la prueba\end{tabular}} & \multicolumn{3}{l|}{Se debe proporcionar una tabla con atributos con valores únicos}                                                                                                                                                                                                                                                                                                                         \\ \hline
			\multicolumn{1}{|l|}{Condiciones}                                                         & \multicolumn{3}{l|}{\begin{tabular}[c]{@{}l@{}}1. Solo se deben proporcionar columnas de tipo \textit{string}\\ 2. Debe estar presente una columna objetivo nominal para la clasificación\end{tabular}}                                                                                                                                                                                                                             \\ \hline
			\multicolumn{4}{|l|}{Combinaciones de valores de entrada}                                                                                                                                                                                                                                                                                                                                                                                                                                                \\ \hline
			\multicolumn{1}{|l|}{CP}                                                                  & \multicolumn{1}{l|}{Escenario}                                                                                                                                      & \multicolumn{1}{l|}{Resultado esperado}                                                                                      & Resultado real                                                                                          \\ \hline
			\multicolumn{1}{|l|}{CP1}                                                                 & \multicolumn{1}{l|}{\begin{tabular}[c]{@{}l@{}}Se proporciona una columna\\ con más del 80\% de valores \\ únicos\end{tabular}}                                     & \multicolumn{1}{l|}{La columna es eliminada}                                                                                 & La columna es eliminada                                                                                 \\ \hline
			\multicolumn{1}{|l|}{CP2}                                                                 & \multicolumn{1}{l|}{\begin{tabular}[c]{@{}l@{}}Se proporciona una columna\\ con varias categorías diferentes\\ donde existan valores sin \\ repetirse\end{tabular}} & \multicolumn{1}{l|}{\begin{tabular}[c]{@{}l@{}}Se reemplazan los valores \\ únicos por la categoría \\ 'other'\end{tabular}} & \begin{tabular}[c]{@{}l@{}}Se reemplazan los valores \\ únicos por la categoría\\  'other'\end{tabular} \\ \hline
		\end{tabular}%
	}
	\end{spacing}
	\caption{Caso de prueba al componente \textit{String preprocs}}
	\label{tab:cp-string-preprocs}
\end{table}

\subsection{Caso de prueba al subcomponente \textit{MV Imputation}}

% Please add the following required packages to your document preamble:
% \usepackage{graphicx}
\begin{table}[H]
	\centering
	\begin{spacing}{1.3}
	\resizebox{\columnwidth}{!}{%
		\begin{tabular}{|llll|}
			\hline
			\multicolumn{4}{|l|}{Caso de prueba}                                                                                                                                                                                                                                                                                                                                                                                                                                 \\ \hline
			\multicolumn{1}{|l|}{\begin{tabular}[c]{@{}l@{}}Objetivo de la \\ prueba\end{tabular}}    & \multicolumn{3}{l|}{Comprobar la efectividad en el tratamiento de valores faltantes en una tabla}                                                                                                                                                                                                                                                                        \\ \hline
			\multicolumn{1}{|l|}{\begin{tabular}[c]{@{}l@{}}Descripción de \\ la prueba\end{tabular}} & \multicolumn{3}{l|}{\begin{tabular}[c]{@{}l@{}}Se debe proporcionar una tabla con valores perdidos al componente \\ \textit{MV Imputation} y evaluar la tabla resultante\end{tabular}}                                                                                                                                                                                            \\ \hline
			\multicolumn{1}{|l|}{Condiciones}                                                         & \multicolumn{3}{l|}{\begin{tabular}[c]{@{}l@{}}1. Debe estar presente la columna objetivo para la clasificación.\\ 2. La columna objetivo debe ser de tipo nominal.\end{tabular}}                                                                                                                                                                                        \\ \hline
			\multicolumn{4}{|l|}{Combinaciones de valores de entrada}                                                                                                                                                                                                                                                                                                                                                                                                            \\ \hline
			\multicolumn{1}{|l|}{CP}                                                                  & \multicolumn{1}{l|}{Escenario}                                                                                   & \multicolumn{1}{l|}{Resultado esperado}                                                                                              & Resultado real                                                                                                 \\ \hline
			\multicolumn{1}{|l|}{CP1}                                                                 & \multicolumn{1}{l|}{\begin{tabular}[c]{@{}l@{}}Se proporciona una \\ tabla con valores\\ faltantes\end{tabular}} & \multicolumn{1}{l|}{\begin{tabular}[c]{@{}l@{}}Se realiza la imputación de\\ valores faltantes con cada\\ método\end{tabular}}       & \begin{tabular}[c]{@{}l@{}}Se realiza la imputación de\\ valores faltantes con cada\\ método\end{tabular}      \\ \hline
			\multicolumn{1}{|l|}{CP2}                                                                 & \multicolumn{1}{l|}{\begin{tabular}[c]{@{}l@{}}Se proporciona una\\ tabla con valores\\ faltantes\end{tabular}}  & \multicolumn{1}{l|}{\begin{tabular}[c]{@{}l@{}}Se comparan los métodos \\ de imputación acorde al \\ algoritmo de ML\end{tabular}}   & \begin{tabular}[c]{@{}l@{}}Se comparan los métodos\\ de imputación acorde al\\ algoritmo de ML\end{tabular}    \\ \hline
			\multicolumn{1}{|l|}{CP3}                                                                 & \multicolumn{1}{l|}{\begin{tabular}[c]{@{}l@{}}Se proporciona una\\ tabla con valores\\ faltantes\end{tabular}}  & \multicolumn{1}{l|}{\begin{tabular}[c]{@{}l@{}}Se devuelven los valores\\ imputados acorde a los \\ resultados del CP2\end{tabular}} & \begin{tabular}[c]{@{}l@{}}Se devuelven los valores\\ imputados acorde a los\\ resultados del CP2\end{tabular} \\ \hline
		\end{tabular}%
	}
	\end{spacing}
	\caption{Caso de prueba al componente \textit{MV Imputation}}
	\label{tab:cp-mv-imp}
\end{table}

\subsection{Caso de prueba al subcomponente \textit{Codificar y normalizar}}

% Please add the following required packages to your document preamble:
% \usepackage{graphicx}
\begin{table}[H]
	\centering
	\begin{spacing}{1.3}
	\resizebox{\columnwidth}{!}{%
		\begin{tabular}{|llll|}
			\hline
			\multicolumn{4}{|l|}{Caso de prueba}                                                                                                                                                                                                                                                                                                                                                                                                                                                     \\ \hline
			\multicolumn{1}{|l|}{\begin{tabular}[c]{@{}l@{}}Objetivo de la\\ prueba\end{tabular}}     & \multicolumn{3}{l|}{Comprobar la efectividad de la codificación y la normalización en una tabla}                                                                                                                                                                                                                                                                                             \\ \hline
			\multicolumn{1}{|l|}{\begin{tabular}[c]{@{}l@{}}Descripción de \\ la prueba\end{tabular}} & \multicolumn{3}{l|}{\begin{tabular}[c]{@{}l@{}}Se debe proporcionar una tabla con valores nominales de alta cardinalidad\\ y numéricos al componente \textit{Codificar y normalizar} y evaluar la tabla resultante.\end{tabular}}     \\ \hline
			\multicolumn{1}{|l|}{Condiciones}                                                         & \multicolumn{3}{l|}{\begin{tabular}[c]{@{}l@{}}1. Debe estar presente la columna objetivo para la clasificación.\\ 2. La columna objetivo debe ser de tipo nominal.\\ 3. Debe haber una columna con más de 15 categorías diferentes.\end{tabular}}                                                                                                                                           \\ \hline
			\multicolumn{4}{|l|}{Combinaciones de valores de entrada}                                                                                                                                                                                                                                                                                                                                                                                                                                \\ \hline
			\multicolumn{1}{|l|}{CP}                                                                  & \multicolumn{1}{l|}{Escenario}                                                                                                  & \multicolumn{1}{l|}{Resultado esperado}                                                                                                 & Resultado real                                                                                                   \\ \hline
			\multicolumn{1}{|l|}{CP1}                                                                 & \multicolumn{1}{l|}{\begin{tabular}[c]{@{}l@{}}Se proporciona una \\ tabla con valores\\ numéricos\end{tabular}}                & \multicolumn{1}{l|}{\begin{tabular}[c]{@{}l@{}}Se realiza la normalización\\ de los datos\end{tabular}}                                 & \begin{tabular}[c]{@{}l@{}}Se realiza la normalización\\ de los datos\end{tabular}                               \\ \hline
			\multicolumn{1}{|l|}{CP2}                                                                 & \multicolumn{1}{l|}{\begin{tabular}[c]{@{}l@{}}Se proporciona una\\ tabla con valores\\ numéricos\end{tabular}}                 & \multicolumn{1}{l|}{\begin{tabular}[c]{@{}l@{}}Se comparan los métodos\\ de normalización acorde\\ al algoritmo de ML\end{tabular}}     & \begin{tabular}[c]{@{}l@{}}Se comparan los métodos\\ de normalización acorde \\ al algoritmo de ML\end{tabular}  \\ \hline
			\multicolumn{1}{|l|}{CP3}                                                                 & \multicolumn{1}{l|}{\begin{tabular}[c]{@{}l@{}}Se proporciona una\\ tabla con valores\\ numéricos\end{tabular}}                 & \multicolumn{1}{l|}{\begin{tabular}[c]{@{}l@{}}Se devuelven los valores\\ normalizados acorde a los \\ resultados del CP2\end{tabular}} & \begin{tabular}[c]{@{}l@{}}Se devuelven los valores\\ normalizados acorde a los\\ resultados del CP2\end{tabular} \\ \hline
			\multicolumn{1}{|l|}{CP4}                                                                 & \multicolumn{1}{l|}{\begin{tabular}[c]{@{}l@{}}Se proporciona una\\ columna con más de 15 \\ categorías distintas\end{tabular}} & \multicolumn{1}{l|}{\begin{tabular}[c]{@{}l@{}}Se realiza la codificación \\ One-Hot a estos valores\end{tabular}}                      & \begin{tabular}[c]{@{}l@{}}Se realiza la codificación\\ One-Hot a estos valores\end{tabular}                     \\ \hline
		\end{tabular}%
	}
	\end{spacing}
	\caption{Caso de prueba al componente \textit{Codificar y normalizar}}
	\label{tab:cp-codificarnorm}
\end{table}

\section{Pruebas de caja negra al componente \textit{AutoML Clasificación (Optimización de Hiperparámetros)}}

\subsection{Caso de prueba para el modelo RProp}

\subsection{Caso de prueba para el modelo PNN}

\subsection{Caso de prueba para el modelo SVM}

\subsection{Caso de prueba para el modelo Random Forest}

\section{Pruebas de integración al componente \textit{AutoML Clasificación (pre-procesado)}}



\section{Conclusiones parciales}
Al terminar este capítulo, se llega a las siguientes conclusiones:
\begin{itemize}
	\item Las pruebas al subcomponente \textit{Discretizer} integrado al \textit{Componente AutoML Clasificación (pre-procesado)}, arrojaron mejores resultados con respecto al componente con una discretización previamente configurada.
	\item El número de intervalos, tras ejecutar las pruebas al \textit{Componente AutoML Clasificación (pre-procesado)}, influyó en los resultados de la clasificación luego de emplear el mismo método de discretización (Equal-width).
	\item Las pruebas individuales al \textit{Componente AutoML Clasificación (Optimización de Hiperparámetros)}, arrojaron mejores porcentajes de acierto con respecto al algoritmo no optimizado, demostrando su correcto funcionamiento.
	\item La prueba de integración del \textit{Componente AutoML Clasificación (Optimización de Hiperparámetros)} al \textit{Componente AutoML Clasificación (pre-procesado)} fue satisfactoria.
	\item Las pruebas realizadas al componente integrado demostraron una mejoría en la clasificación en comparación con el no integrado.
\end{itemize}
\pagebreak

	
	
    \cleardoublepage
    \phantomsection
	\addcontentsline{toc}{chapter}{Conclusiones}
    \fancyhf{}
    \lhead[\thepage]{\textbf{Conclusiones}}
    \rhead[\textbf{Conclusiones}]{\thepage}
    \chapter*{Conclusiones generales}
Con el cumplimiento de los objetivos planteados para la investigación, se puede llegar a las siguientes conclusiones:
\begin{itemize}
	\item KNIME se ha identificado como una herramienta valiosa para la minería de datos, permitiendo la integración de H2O y la implementación de \textit{AutoML}. Sin embargo, tanto el componente \textit{AutoML} de KNIME como el componente \textit{AutoML Clasificación (pre-procesado)} de la CUJAE presentan áreas de mejora, especialmente en la fase de pre-procesado y la optimización de hiperparámetros.
	\item La disponibilidad de documentación y la posibilidad de modificación son factores clave a considerar para el desarrollo de esta investigación, por lo que se elige el componente \textit{AutoML Clasificación (pre-procesado)}.
	\item Las técnicas de pre-procesado como discretización, normalización y tratamiento de valores faltantes, a pesar de tener una gran variedad de métodos cada una, se ha demostrado que su efectividad depende del conjunto de datos y modelo de Aprendizaje Automático que se vaya a emplear.
	\item Las métricas Exactitud y Cohen's Kappa para la evaluación de los modelos permiten obtener una visión general del rendimiento de estos, independientemente de si los conjuntos de datos son de naturaleza binaria o multiclase.
	\item El impacto de las técnicas de pre-procesado puede variar dependiendo del modelo y del contexto del conjunto de datos: mientras que el modelo ID3 mostró una mejora sistemática con la integración de pre-procesado, el modelo C4.5 presentó resultados inferiores en uno de los experimentos.
	\item El modelo que obtuvo los mejores resultados, de forma general, fue Random Forest.
	\item El modelo que obtuvo los peores resultados, de forma general, fue SVM sin la integración de los nuevos componentes. Sin embargo, al realizarse la integración con estas implementaciones, mejora en un promedio de 25.6\% en términos de exactitud.
	\item Con las nuevas integraciones, el modelo que presentó una mejora mayor en términos de exactitud es PNN, con un incremento de un 27.1\% de promedio.
	\item Entre las métricas evaluadas en la experimentación, el Cohen’s Kappa mostró el mayor incremento con la integración de los componentes de pre-procesado y optimización de hiperparámetros, con un promedio de 25.69\%.
\end{itemize}
    
    \cleardoublepage
    \phantomsection
	\addcontentsline{toc}{chapter}{Recomendaciones}
	\fancyhf{}
	\lhead[\thepage]{\textbf{Recomendaciones}}
	\rhead[\textbf{Recomendaciones}]{\thepage}
	\chapter*{Recomendaciones}
\begin{itemize}
	\item Ampliar la investigación enfocándose en la actualización del componente \textit{AutoML} de KNIME.
	\item Extender el conjunto de modelos disponibles a entrenar, sin delimitarse a la tarea de clasificación.
	\item Desarrollar pre-procesado para el tratamiento de fechas y datos de tiempo.
\end{itemize}

	
	\cleardoublepage
	\phantomsection
	\addcontentsline{toc}{chapter}{Referencias bibliográficas}
	\fancyhf{}
	\lhead[\thepage]{\textbf{Referencias bibliográficas}}
	\rhead[\textbf{Referencias bibliográficas}]{\thepage}
	\bibliographystyle{authordate1}
	\bibliography{bibliografia-pp2}

	
	\cleardoublepage
	\phantomsection
%	\addcontentsline{toc}{chapter}{Anexos}
	\fancyhf{}
	\lhead[\thepage]{\textbf{Anexos}}
	\rhead[\textbf{Anexos}]{\thepage}
	\appendix
\clearpage{\renewcommand{\appendixname}{Anexo}

\chapter{Anexos}\label{anex}




	
\end{document}