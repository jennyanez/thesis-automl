\chapter*{Conclusiones generales}
Con el cumplimiento de los objetivos planteados para la investigación, se puede llegar a las siguientes conclusiones:
\begin{itemize}
	\item El proceso de descubrimiento de conocimiento en bases de datos (KDD) es un proceso iterativo que consiste en varias etapas, incluyendo la selección de datos, la limpieza de datos, la transformación de datos y la minería de datos.
	\item La Minería de Datos es el proceso de descubrir patrones y relaciones interesantes en grandes conjuntos de datos, utilizando técnicas de aprendizaje automático, estadísticas y visualización de datos.
	\item El Aprendizaje Automático es una técnica que permite a las computadoras aprender a partir de datos, sin necesidad de ser programadas explícitamente para ello.
	\item Las principales tareas del AutoML son la selección de características, selección de modelos, pre-procesado de datos, ajuste de hiperparámetros y evaluación e interpretación de modelos.
	\item KNIME presenta pocas opciones disponibles para soportar AutoML, sobre todo para la etapa de pre-procesado y la optimización de hiperparámetros.
	\item Se implementó un subcomponente para la discretización automática de variables numéricas, basado en el rendimiento del algoritmo de clasificacion ID3.
	\item Se implementó un componente que brinda soporte para tareas de AutoML, enfocándose en la etapa de optimización de hiperparámetros.
	\item Se demostró el correcto funcionamiento del componente propuesto y los subcomponentes que lo integran.
\end{itemize}