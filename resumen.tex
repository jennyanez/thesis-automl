\section*{Resumen}
En un contexto donde la generación masiva de datos desafía la capacidad de convertirlos en información valiosa para la toma de decisiones, el Aprendizaje Automático emerge como una herramienta esencial. Este enfoque se centra en el desarrollo de modelos y algoritmos que capacitan a las computadoras para aprender de los datos sin ser programadas explícitamente para hacerlo. Sin embargo, la efectiva implementación de estas técnicas requiere intervención humana, y la Automatización del Aprendizaje Automático (\textit{AutoML}) ha evolucionado como solución para simplificar y agilizar este proceso. \\
KNIME, una herramienta versátil, facilita la implementación de \textit{AutoML} a través de la integración con la plataforma H2O y dos componentes \textit{AutoML}: la extensión H2O, con nodos específicos para tareas concretas, y el componente \textit{AutoML Learner}, desarrollado por el equipo de KNIME, que automatiza la selección del mejor algoritmo y ajuste de hiperparámetros. Aunque ofrece automatización, carece de la capacidad de control directo y ajuste manual de los procesos, limitando así la personalización de modelos. Por otra parte, se encuentra el componente \textit{AutoML Clasificación (pre-procesado)}, desarrollado en la CUJAE \citep{Carrazana2022}, que aborda la ejecución y comparación de flujos de \textit{AutoML} en tareas de clasificación con componentes enfocados en tareas concisas de pre-procesamiento. Sin embargo, la ausencia de optimización de hiperparámetros y la falta de automatización en actividades esenciales, como discretización, normalización, manejo de valores faltantes y tratamiento de valores de alta cardinalidad, plantean desafíos en el procesamiento de datos y la precisión de los modelos de Aprendizaje Automático. \\
La presente investigación aborda esta brecha proponiendo una nueva versión del componente \textit{AutoML Clasificación (pre-procesado)}, donde se crean componentes para el pre-procesado y la optimización de hiperparámetros que, integrándolos con el componente \textit{AutoML Clasificación (pre-procesado)}, conforman el componente \textit{AutoML Clasificación}. Se demuestra a través de su evaluación con cinco experimentos, empleando cinco conjuntos de datos con características diferentes, cómo con las nuevas implementaciones se logra una mejoría en el rendimiento de los modelos, lo que evidencia la capacidad del sistema para adaptarse a diferentes conjuntos de datos y necesidades específicas de clasificación.
 

\begin{description}
	\item[Palabras clave:]{Aprendizaje Automático, Minería de datos, AutoML, KNIME, optimización de hiperparámetros, pre-procesamiento de datos, clasificación.}
\end{description}
%\end{abstract}


