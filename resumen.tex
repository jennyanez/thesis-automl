\section*{Resumen}
Vivimos en un mundo en el que se generan grandes cantidades de datos, los cuales se almacenan en diferentes sistemas, y el reto es convertir esos datos en información útil para la toma de decisiones. Una técnica para extraer información valiosa de grandes cantidades de datos es el Aprendizaje Automático, que se enfoca en el desarrollo de modelos y algoritmos que permiten a las computadoras aprender de los datos sin ser programadas explícitamente para hacerlo. Para implementar efectivamente estas técnicas, se requiere de la intervención humana, y la automatización del aprendizaje automático (AutoML) se ha desarrollado como una solución para simplificar y acelerar este proceso. Para ello existen numerosas herramientas, como KNIME, que permite la implementación de AutoML a través de diferentes nodos. En \citep{Carrazana2022} se desarrolla un componente dedicado a esta tarea, específicamente para el pre-procesado en tareas de clasificación. No obstante, quedaron tareas pendientes, como la optimización de hiperparámetros y la automatización de tareas en la fase de transformación de los datos, las cuales son implementadas en la presente investigación. 



\begin{description}
	\item[Palabras clave:]{Aprendizaje Automático, Minería de datos, AutoML, KNIME, optimización de hiperparámetros, preprocesamiento de datos, clasificación.}
\end{description}
%\end{abstract}


