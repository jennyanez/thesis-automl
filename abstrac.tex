\section*{Abstract}
In a context where the massive generation of data challenges the ability to turn it into valuable information for decision-making, Machine Learning emerges as an essential tool. This approach focuses on developing models and algorithms that enable computers to learn from data without being explicitly programmed to do so. However, the effective implementation of these techniques requires human intervention, and Automated Machine Learning (AutoML) has evolved as a solution to simplify and expedite this process.\\
KNIME, a versatile tool, facilitates the implementation of AutoML through integration with the H2O platform and two AutoML components: the H2O extension, with specific nodes for concrete tasks, and the AutoML Learner component, developed by the KNIME team, which automates the selection of the best algorithm and hyperparameter tuning. Although it offers automation, it lacks the capability for direct control and manual adjustment of processes, thus limiting model customization. On the other hand, there is the AutoML Classification (pre-processing) component, developed at CUJAE by Carrazana (2022), addressing the execution and comparison of AutoML flows in classification tasks with components focused on concise pre-processing tasks. However, the absence of hyperparameter optimization and the lack of automation in essential activities, such as discretization, normalization, handling missing values, and treatment of high cardinality values, pose challenges in data processing and the accuracy of Machine Learning models. \\
This research addresses this gap by proposing a new version of the AutoML Classification (pre-processing) component, where components are created for pre-processing and hyperparameter optimization that, when integrated with the AutoML Classification (pre-processing) component, form the AutoML Classification component. It is demonstrated through its evaluation with five experiments, using five datasets with different characteristics, how with the new implementations an improvement in the performance of the models is achieved, which evidences the system's ability to adapt to different datasets and specific classification needs.

\begin{description}
	\item[Keywords:]{Machine Learning, Data Mining, AutoML, KNIME, HPO, data pre-processing, classification, HPO, hiperparameter optimization}
\end{description}
%\end{abstract}

