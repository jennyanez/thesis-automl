\chapter*{Introducción}
En la actualidad, en el mundo se generan cada vez más datos y se almacenan en diversos tipos de sistemas, lo que nos presenta un gran desafío: ¿cómo convertir estos datos en información valiosa que pueda ser utilizada para tomar decisiones informadas? Para resolver este reto, se han desarrollado diversas técnicas y herramientas para extraer información útil de grandes cantidades de datos. \\
El proceso de descubrimiento de conocimiento en bases de datos (\textit{KDD}, por sus siglas en inglés), es un procedimiento que se utiliza para extraer conocimiento útil y relevante, a partir de grandes cantidades de datos almacenados en diversos sistemas \citep{orallo2004}. Este consta de varias fases: integración y recopilación; selección, limpieza y transformación; aplicación de algoritmos de minería de datos; evaluación e interpretación; así como la difusión y uso del conocimiento obtenido \citep{Han2011}. El proceso de descubrimiento de conocimiento en bases de datos ha sentado las bases para una disciplina relacionada, conocida como minería de datos. \\
La minería de datos es el proceso de extraer conocimiento útil y comprensible, previamente desconocido, desde grandes cantidades de datos almacenados en distintos formatos \citep{orallo2004}. Una de las técnicas más comunes de la minería de datos es la clasificación, que consiste en la identificación de un conjunto de categorías o etiquetas para un conjunto de datos no etiquetados \citep{orallo2004}. \\
La clasificación es una técnica muy utilizada en diferentes áreas, como la detección de spam en el correo electrónico \citep{mendez2007sistemas}, la clasificación de imágenes \citep{borras2017clasificacion} y la identificación de transacciones fraudulentas en tarjetas de crédito \citep{dhankhad2018supervised}. Al utilizar algoritmos de clasificación es posible predecir la categoría a la que pertenece un nuevo conjunto de datos, lo que puede ser de gran utilidad en la toma de decisiones y la mejora de los procesos. Esta tarea, a su vez, es uno de los principales enfoques del Aprendizaje Automático (Machine Learning), una disciplina dentro de la inteligencia artificial, que se basa en la idea de que las computadoras pueden aprender a reconocer patrones y tomar decisiones precisas y acertadas, a través de la experiencia y la retroalimentación continua de los datos. \\
El Aprendizaje Automático se enfoca en el desarrollo de algoritmos y modelos que permiten a las computadoras aprender de los datos sin ser programadas explícitamente para hacerlo. Se define como un conjunto de métodos que puede detectar patrones en los datos automáticamente, y luego usar los patrones descubiertos para predecir datos en el futuro, o para realizar otro tipo de toma de decisiones bajo incertidumbre \citep{murphy2012machine}. Estos patrones interesantes son los que representan el conocimiento. La implementación efectiva de estas técnicas requiere la intervención humana, incluyendo la selección de algoritmos adecuados, el pre-procesamiento de datos y la optimización de hiperparámetros. La Automatización del Aprendizaje Automático (AutoML) se ha desarrollado como una solución para simplificar y acelerar este proceso. \\
 AutoML tiene como objetivo tomar estas decisiones de una manera automatizada, objetiva y basada en datos: el usuario simplemente proporciona datos y el sistema AutoML determina automáticamente el enfoque que funciona mejor para esta aplicación en particular \citep{hutter2019automated}. Para realizar esta tarea de manera efectiva, se requiere de herramientas especializadas que permitan procesar grandes cantidades de información de forma rápida y eficiente. Una de estas herramientas es KNIME.\\
KNIME es una herramienta popular que proporciona un entorno de desarrollo visual para la creación, ejecución y evaluación de flujos de trabajo de análisis de datos. Es una plataforma de software libre y abierto que incluye una amplia variedad de nodos para el pre-procesado de datos, la minería de datos y la modelización de datos \citep{knime2023}. En KNIME, el AutoML se puede implementar a través de una extensión de H2O y dos componentes AutoML; la extensión de H2O, que acoge múltiples nodos para tareas concretas; y el componente \textit{AutoML Learner}, que permite seleccionar automáticamente el mejor algoritmo de aprendizaje automático para un conjunto de datos en particular, así como ajustar los hiperparámetros del modelo. No obstante, una de las principales desventajas que presenta es la falta de control y personalización. Aunque permite a los usuarios seleccionar automáticamente los algoritmos, no tienen control directo sobre estos procesos y no pueden ajustar los parámetros de manera manual. Esto puede limitar la capacidad de los usuarios para personalizar y optimizar los modelos para satisfacer sus necesidades específicas. Mientras, el componente \textit{AutoML (Componente AutoML Clasificación (pre-procesado)} \citep{Carrazana2022}, desarrollado en la CUJAE, ejecuta y compara el desempeño de múltiples flujos de AutoML en tareas de clasificación. En este componente se desarrollaron subcomponentes enfocados en tareas concisas de pre-procesado. Sin embargo, no contempla la optimización de hiperparámetros y la automatización de algunas de las actividades esenciales en el pre-procesado, como discretización y normalización, quedaron pendientes, lo que dificulta el procesamiento de datos y la precisión de los modelos de Aprendizaje Automático. De esta manera se genera un \textbf{problema}: la inexistencia de un componente en KNIME para AutoML que contenga un pre-procesado con las tareas automatizadas e incorpore la optimización de hiperparámetros. \\
En aras de resolver la problemática planteada se propone una nueva versión del componente \textit{AutoML (Componente AutoML Clasificación (pre-procesado)} \citep{Carrazana2022}, dado que este permite modificaciones, a diferencia del resto. Por tanto, se determina como \textbf{objetivo general} desarrollar una nueva versión del componente KNIME que permita automatizar tareas en el pre-procesado y la selección de hiperparámetros. Este objetivo general se desglosa en los siguientes \textbf{objetivos específicos y tareas}:

\begin{enumerate}
	\item Analizar el estado del arte de las principales técnicas de Aprendizaje Automático Automatizado para el pre-procesado y la optimización de hiperparámetros.
	\begin{enumerate}
		\item Describir las características del proceso KDD y Minería de Datos. 
		\item Caracterizar las técnicas de Aprendizaje Automático y las principales tareas de Automatización del Aprendizaje Automático.
		\item Asimilar componente AutoML Clasificación (pre-procesado).
	\end{enumerate}
	\item Desarrollar flujos de AutoML en KNIME en tareas de clasificación.
	\begin{enumerate}
		\item Desarrollar subcomponentes en KNIME para la automatización de actividades en el pre-procesado. (discretización, normalización, tratamiento de valores faltantes y de valores únicos).
		\item Desarrollar un componente KNIME para la optimización de hiperparámetros.
	\end{enumerate}
	\item Validar subcomponentes de actividades del pre-procesado de datos y componente AutoML Clasificación (Optimización de Hiperparámetros).
	\begin{enumerate}
		\item Desarrollar los casos de pruebas que permitan validar los subcomponentes de pre-procesado de datos propuestos. 
		\item Desarrollar los casos de pruebas que permitan validar el componente AutoML Clasificación (Optimización de Hiperparámetros).
		\item Evaluar los resultados arrojados por los casos de prueba.
	\end{enumerate} 
	\item Validar integración al componente AutoML Clasificación (pre-procesado)
	\begin{enumerate}
		\item Diseñar y ejecutar los casos de pruebas que permitan validar la integración de los subcomponentes de pre-procesado de datos y el componente AutoML Clasificación (Optimización de Hiperparámetros) al componente AutoML Clasificación (pre-procesado).
		\item Evaluar los resultados arrojados por los casos de prueba.
	\end{enumerate}
\end{enumerate}

%El \textbf{valor práctico} de este proyecto consiste en las modificaciones desarrolladas al componente KNIME de AutoML para el pre-procesado en tareas de clasificación, capaz de automatizar la optimización de hiperparámetros y ejecutar flujos de pre-procesado para diferentes algoritmos en dicha tarea. \\
En cuanto a la estructuración, este trabajo está dividido en tres capítulos:
\begin{itemize}
	\item \textbf{Capítulo 1: Aprendizaje Automático y AutoML}, se presenta el estudio realizado sobre las temáticas que aborda el trabajo, se muestran los conceptos fundamentales relacionados con el Aprendizaje Automático, la Minería de Datos y AutoML; así como la descripción del componente KNIME de AutoML para pre-procesado.
	\item \textbf{Capítulo 2: Propuesta de modificación al componente KNIME de AutoML para pre-procesado}, se presenta y expone el diseño e implementación de la modificación al componente KNIME para tareas de AutoML en pre-procesado y optimización de hiperparámetros.
		\item \textbf{Capítulo 3: Integración y validación de soluciones propuestas al componente de AutoML}, se muestran, comparan y analizan los resultados obtenidos de los algoritmos para diferentes configuraciones.
\end{itemize}




%Las figuras deben referenciarse en el texto, así como lo muestra este ejemplo, ver Figura \ref{fig:figCUJAE}.

%\begin{figure}[H] %la opción H indica al compilador LaTeX que posicione la figura lo más cerca posible de este lugar.
%\centering
 % \includegraphics[width=0.5\linewidth]{figuras/membrete-cujae-centrado.png}
 % \caption{El título de la figura debe estar acorde con su contenido.}
 % \label{fig:figCUJAE} %incluir el label permite referenciarla en cualquier parte del documento.
%\end{figure}

%Se deben utilizar siempre los mismos términos para referirse a los mismos conceptos y no olvidar de definir los términos que son claves en el campo de acción, o sea, la propuesta de la tesis.\\

%Describir la Situación problemática. Problema. Objetivo general. Objetivos específicos. Tareas. Beneficios. Breve resumen del contenido de la tesis.