\chapter{Validación de soluciones propuestas al componente de AutoML}\label{chap:3}
En este capítulo .....


\section{Pruebas de caja negra a subcomponentes para el pre-procesado}

... muelita linda

\subsection{Caso de prueba al subcomponente \textit{Discretizer}}

% Please add the following required packages to your document preamble:
% \usepackage{graphicx}
\begin{table}[H]
	\centering
	\begin{spacing}{1.3}
	\resizebox{\columnwidth}{!}{%
		\begin{tabular}{|llll|}
			\hline
			\multicolumn{4}{|l|}{Caso de prueba}                                                                                                                                                                                                                                                                                                                                                                                                                                         \\ \hline
			\multicolumn{1}{|l|}{\begin{tabular}[c]{@{}l@{}}Objetivo de la \\ prueba\end{tabular}}    & \multicolumn{3}{l|}{Comprobar la efectividad de las transformaciones al discretizar las variables numéricas}                                                                                                                                                                                                                                                                     \\ \hline
			\multicolumn{1}{|l|}{\begin{tabular}[c]{@{}l@{}}Descripción de \\ la prueba\end{tabular}} & \multicolumn{3}{l|}{\begin{tabular}[c]{@{}l@{}}Se debe proporcionar una tabla con variables numéricas al componente \textit{Discretizer} y \\ evaluar la tabla resultante\end{tabular}}                                                                                                                                                                                                   \\ \hline
			\multicolumn{1}{|l|}{Condiciones}                                                         & \multicolumn{3}{l|}{\begin{tabular}[c]{@{}l@{}}1. Debe estar presente la columna objetivo para la clasificación.\\ 2. La columna objetivo debe ser de tipo nominal.\end{tabular}}                                                                                                                                                                                                \\ \hline
			\multicolumn{4}{|l|}{Combinaciones de valores de entrada}                                                                                                                                                                                                                                                                                                                                                                                                                    \\ \hline
			\multicolumn{1}{|l|}{CP}                                                                  & \multicolumn{1}{l|}{Escenario}                                                                                     & \multicolumn{1}{l|}{Resultado esperado}                                                                                                 & Resultado real                                                                                                    \\ \hline
			\multicolumn{1}{|l|}{CP1}                                                                 & \multicolumn{1}{l|}{\begin{tabular}[c]{@{}l@{}}Se proporciona una \\ tabla con atributos\\ numéricos\end{tabular}} & \multicolumn{1}{l|}{\begin{tabular}[c]{@{}l@{}}Se discretizan los atributos \\ numéricos con cada método\end{tabular}}                  & \begin{tabular}[c]{@{}l@{}}Se discretizan los atributos \\ numéricos con cada método\end{tabular}                 \\ \hline
			\multicolumn{1}{|l|}{CP2}                                                                 & \multicolumn{1}{l|}{\begin{tabular}[c]{@{}l@{}}Se proporciona una\\ tabla con atributos\\ numéricos\end{tabular}}  & \multicolumn{1}{l|}{\begin{tabular}[c]{@{}l@{}}Se comparan los discretizadores \\ acorde al algoritmo de ML\end{tabular}}               & \begin{tabular}[c]{@{}l@{}}Se comparan los discretizadores \\ acorde al algoritmo de ML\end{tabular}              \\ \hline
			\multicolumn{1}{|l|}{CP3}                                                                 & \multicolumn{1}{l|}{\begin{tabular}[c]{@{}l@{}}Se proporciona una\\ tabla con atributos \\ numéricos\end{tabular}} & \multicolumn{1}{l|}{\begin{tabular}[c]{@{}l@{}}Se devuelven los datos \\ discretizados acorde a los \\ resultados del CP2\end{tabular}} & \begin{tabular}[c]{@{}l@{}}Se devuelven los datos\\ discretizados acorde a los \\ resultados del CP2\end{tabular} \\ \hline
		\end{tabular}%
	}
	\end{spacing}
	\caption{Caso de prueba al componente \textit{Discretizer}}
	\label{tab:cp-disc}
\end{table}



\subsection{Caso de prueba al subcomponente \textit{String preprocs}}

% Please add the following required packages to your document preamble:
% \usepackage{graphicx}
\begin{table}[H]
	\centering
	\begin{spacing}{1.3}
	\resizebox{\columnwidth}{!}{%
		\begin{tabular}{|llll|}
			\hline
			\multicolumn{4}{|l|}{Caso de prueba}                                                                                                                                                                                                                                                                                                                                                                                                                                                                     \\ \hline
			\multicolumn{1}{|l|}{\begin{tabular}[c]{@{}l@{}}Objetivo de la \\ prueba\end{tabular}}    & \multicolumn{3}{l|}{\begin{tabular}[c]{@{}l@{}}Comprobar la efectividad de las transformaciones a las columnas de tipo \textit{string}\\ y evaluar la tabla resultante\end{tabular}}                                                                                                                                                                                                                                  \\ \hline
			\multicolumn{1}{|l|}{\begin{tabular}[c]{@{}l@{}}Descripción de \\ la prueba\end{tabular}} & \multicolumn{3}{l|}{Se debe proporcionar una tabla con atributos con valores únicos}                                                                                                                                                                                                                                                                                                                         \\ \hline
			\multicolumn{1}{|l|}{Condiciones}                                                         & \multicolumn{3}{l|}{\begin{tabular}[c]{@{}l@{}}1. Solo se deben proporcionar columnas de tipo \textit{string}\\ 2. Debe estar presente una columna objetivo nominal para la clasificación\end{tabular}}                                                                                                                                                                                                                             \\ \hline
			\multicolumn{4}{|l|}{Combinaciones de valores de entrada}                                                                                                                                                                                                                                                                                                                                                                                                                                                \\ \hline
			\multicolumn{1}{|l|}{CP}                                                                  & \multicolumn{1}{l|}{Escenario}                                                                                                                                      & \multicolumn{1}{l|}{Resultado esperado}                                                                                      & Resultado real                                                                                          \\ \hline
			\multicolumn{1}{|l|}{CP1}                                                                 & \multicolumn{1}{l|}{\begin{tabular}[c]{@{}l@{}}Se proporciona una columna\\ con más del 80\% de valores \\ únicos\end{tabular}}                                     & \multicolumn{1}{l|}{La columna es eliminada}                                                                                 & La columna es eliminada                                                                                 \\ \hline
			\multicolumn{1}{|l|}{CP2}                                                                 & \multicolumn{1}{l|}{\begin{tabular}[c]{@{}l@{}}Se proporciona una columna\\ con varias categorías diferentes\\ donde existan valores sin \\ repetirse\end{tabular}} & \multicolumn{1}{l|}{\begin{tabular}[c]{@{}l@{}}Se reemplazan los valores \\ únicos por la categoría \\ 'other'\end{tabular}} & \begin{tabular}[c]{@{}l@{}}Se reemplazan los valores \\ únicos por la categoría\\  'other'\end{tabular} \\ \hline
		\end{tabular}%
	}
	\end{spacing}
	\caption{Caso de prueba al componente \textit{String preprocs}}
	\label{tab:cp-string-preprocs}
\end{table}

\subsection{Caso de prueba al subcomponente \textit{MV Imputation}}

% Please add the following required packages to your document preamble:
% \usepackage{graphicx}
\begin{table}[H]
	\centering
	\begin{spacing}{1.3}
	\resizebox{\columnwidth}{!}{%
		\begin{tabular}{|llll|}
			\hline
			\multicolumn{4}{|l|}{Caso de prueba}                                                                                                                                                                                                                                                                                                                                                                                                                                 \\ \hline
			\multicolumn{1}{|l|}{\begin{tabular}[c]{@{}l@{}}Objetivo de la \\ prueba\end{tabular}}    & \multicolumn{3}{l|}{Comprobar la efectividad en el tratamiento de valores faltantes en una tabla}                                                                                                                                                                                                                                                                        \\ \hline
			\multicolumn{1}{|l|}{\begin{tabular}[c]{@{}l@{}}Descripción de \\ la prueba\end{tabular}} & \multicolumn{3}{l|}{\begin{tabular}[c]{@{}l@{}}Se debe proporcionar una tabla con valores perdidos al componente \\ \textit{MV Imputation} y evaluar la tabla resultante\end{tabular}}                                                                                                                                                                                            \\ \hline
			\multicolumn{1}{|l|}{Condiciones}                                                         & \multicolumn{3}{l|}{\begin{tabular}[c]{@{}l@{}}1. Debe estar presente la columna objetivo para la clasificación.\\ 2. La columna objetivo debe ser de tipo nominal.\end{tabular}}                                                                                                                                                                                        \\ \hline
			\multicolumn{4}{|l|}{Combinaciones de valores de entrada}                                                                                                                                                                                                                                                                                                                                                                                                            \\ \hline
			\multicolumn{1}{|l|}{CP}                                                                  & \multicolumn{1}{l|}{Escenario}                                                                                   & \multicolumn{1}{l|}{Resultado esperado}                                                                                              & Resultado real                                                                                                 \\ \hline
			\multicolumn{1}{|l|}{CP1}                                                                 & \multicolumn{1}{l|}{\begin{tabular}[c]{@{}l@{}}Se proporciona una \\ tabla con valores\\ faltantes\end{tabular}} & \multicolumn{1}{l|}{\begin{tabular}[c]{@{}l@{}}Se realiza la imputación de\\ valores faltantes con cada\\ método\end{tabular}}       & \begin{tabular}[c]{@{}l@{}}Se realiza la imputación de\\ valores faltantes con cada\\ método\end{tabular}      \\ \hline
			\multicolumn{1}{|l|}{CP2}                                                                 & \multicolumn{1}{l|}{\begin{tabular}[c]{@{}l@{}}Se proporciona una\\ tabla con valores\\ faltantes\end{tabular}}  & \multicolumn{1}{l|}{\begin{tabular}[c]{@{}l@{}}Se comparan los métodos \\ de imputación acorde al \\ algoritmo de ML\end{tabular}}   & \begin{tabular}[c]{@{}l@{}}Se comparan los métodos\\ de imputación acorde al\\ algoritmo de ML\end{tabular}    \\ \hline
			\multicolumn{1}{|l|}{CP3}                                                                 & \multicolumn{1}{l|}{\begin{tabular}[c]{@{}l@{}}Se proporciona una\\ tabla con valores\\ faltantes\end{tabular}}  & \multicolumn{1}{l|}{\begin{tabular}[c]{@{}l@{}}Se devuelven los valores\\ imputados acorde a los \\ resultados del CP2\end{tabular}} & \begin{tabular}[c]{@{}l@{}}Se devuelven los valores\\ imputados acorde a los\\ resultados del CP2\end{tabular} \\ \hline
		\end{tabular}%
	}
	\end{spacing}
	\caption{Caso de prueba al componente \textit{MV Imputation}}
	\label{tab:cp-mv-imp}
\end{table}

\subsection{Caso de prueba al subcomponente \textit{Codificar y normalizar}}

% Please add the following required packages to your document preamble:
% \usepackage{graphicx}
\begin{table}[H]
	\centering
	\begin{spacing}{1.3}
	\resizebox{\columnwidth}{!}{%
		\begin{tabular}{|llll|}
			\hline
			\multicolumn{4}{|l|}{Caso de prueba}                                                                                                                                                                                                                                                                                                                                                                                                                                                     \\ \hline
			\multicolumn{1}{|l|}{\begin{tabular}[c]{@{}l@{}}Objetivo de la\\ prueba\end{tabular}}     & \multicolumn{3}{l|}{Comprobar la efectividad de la codificación y la normalización en una tabla}                                                                                                                                                                                                                                                                                             \\ \hline
			\multicolumn{1}{|l|}{\begin{tabular}[c]{@{}l@{}}Descripción de \\ la prueba\end{tabular}} & \multicolumn{3}{l|}{\begin{tabular}[c]{@{}l@{}}Se debe proporcionar una tabla con valores nominales de alta cardinalidad\\ y numéricos al componente \textit{Codificar y normalizar} y evaluar la tabla resultante.\end{tabular}}     \\ \hline
			\multicolumn{1}{|l|}{Condiciones}                                                         & \multicolumn{3}{l|}{\begin{tabular}[c]{@{}l@{}}1. Debe estar presente la columna objetivo para la clasificación.\\ 2. La columna objetivo debe ser de tipo nominal.\\ 3. Debe haber una columna con más de 15 categorías diferentes.\end{tabular}}                                                                                                                                           \\ \hline
			\multicolumn{4}{|l|}{Combinaciones de valores de entrada}                                                                                                                                                                                                                                                                                                                                                                                                                                \\ \hline
			\multicolumn{1}{|l|}{CP}                                                                  & \multicolumn{1}{l|}{Escenario}                                                                                                  & \multicolumn{1}{l|}{Resultado esperado}                                                                                                 & Resultado real                                                                                                   \\ \hline
			\multicolumn{1}{|l|}{CP1}                                                                 & \multicolumn{1}{l|}{\begin{tabular}[c]{@{}l@{}}Se proporciona una \\ tabla con valores\\ numéricos\end{tabular}}                & \multicolumn{1}{l|}{\begin{tabular}[c]{@{}l@{}}Se realiza la normalización\\ de los datos\end{tabular}}                                 & \begin{tabular}[c]{@{}l@{}}Se realiza la normalización\\ de los datos\end{tabular}                               \\ \hline
			\multicolumn{1}{|l|}{CP2}                                                                 & \multicolumn{1}{l|}{\begin{tabular}[c]{@{}l@{}}Se proporciona una\\ tabla con valores\\ numéricos\end{tabular}}                 & \multicolumn{1}{l|}{\begin{tabular}[c]{@{}l@{}}Se comparan los métodos\\ de normalización acorde\\ al algoritmo de ML\end{tabular}}     & \begin{tabular}[c]{@{}l@{}}Se comparan los métodos\\ de normalización acorde \\ al algoritmo de ML\end{tabular}  \\ \hline
			\multicolumn{1}{|l|}{CP3}                                                                 & \multicolumn{1}{l|}{\begin{tabular}[c]{@{}l@{}}Se proporciona una\\ tabla con valores\\ numéricos\end{tabular}}                 & \multicolumn{1}{l|}{\begin{tabular}[c]{@{}l@{}}Se devuelven los valores\\ normalizados acorde a los \\ resultados del CP2\end{tabular}} & \begin{tabular}[c]{@{}l@{}}Se devuelven los valores\\ normalizados acorde a los\\ resultados del CP2\end{tabular} \\ \hline
			\multicolumn{1}{|l|}{CP4}                                                                 & \multicolumn{1}{l|}{\begin{tabular}[c]{@{}l@{}}Se proporciona una\\ columna con más de 15 \\ categorías distintas\end{tabular}} & \multicolumn{1}{l|}{\begin{tabular}[c]{@{}l@{}}Se realiza la codificación \\ One-Hot a estos valores\end{tabular}}                      & \begin{tabular}[c]{@{}l@{}}Se realiza la codificación\\ One-Hot a estos valores\end{tabular}                     \\ \hline
		\end{tabular}%
	}
	\end{spacing}
	\caption{Caso de prueba al componente \textit{Codificar y normalizar}}
	\label{tab:cp-codificarnorm}
\end{table}

\section{Pruebas de caja negra al componente \textit{AutoML Clasificación (Optimización de Hiperparámetros)}}

\subsection{Caso de prueba para el modelo RProp}

\subsection{Caso de prueba para el modelo PNN}

\subsection{Caso de prueba para el modelo SVM}

\subsection{Caso de prueba para el modelo Random Forest}

\section{Pruebas de integración al componente \textit{AutoML Clasificación (pre-procesado)}}



\section{Conclusiones parciales}
Al terminar este capítulo, se llega a las siguientes conclusiones:
\begin{itemize}
	\item Las pruebas al subcomponente \textit{Discretizer} integrado al \textit{Componente AutoML Clasificación (pre-procesado)}, arrojaron mejores resultados con respecto al componente con una discretización previamente configurada.
	\item El número de intervalos, tras ejecutar las pruebas al \textit{Componente AutoML Clasificación (pre-procesado)}, influyó en los resultados de la clasificación luego de emplear el mismo método de discretización (Equal-width).
	\item Las pruebas individuales al \textit{Componente AutoML Clasificación (Optimización de Hiperparámetros)}, arrojaron mejores porcentajes de acierto con respecto al algoritmo no optimizado, demostrando su correcto funcionamiento.
	\item La prueba de integración del \textit{Componente AutoML Clasificación (Optimización de Hiperparámetros)} al \textit{Componente AutoML Clasificación (pre-procesado)} fue satisfactoria.
	\item Las pruebas realizadas al componente integrado demostraron una mejoría en la clasificación en comparación con el no integrado.
\end{itemize}
\pagebreak
